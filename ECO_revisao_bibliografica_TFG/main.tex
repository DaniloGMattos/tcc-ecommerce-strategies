% O plano deve conter até 04 (quatro) páginas, excetuando-se a capa e incluindo-se as referências,
% apenas em formato “.doc” ou “.pdf”, nas seguintes especificações obrigatórias:
% OK * Formato A4, margens superior 1,5 cm; inferior 2,5 cm; esquerda e direita 2,0 cm;
% OK * Parágrafos com espaçamento: 0 pt (Antes), 6 pt (Depois) e 1,5 linha (Entre linhas);
% OK * Fonte dos títulos e subtítulos: Arial 12, negrito, alinhamento à esquerda;
% OK * Fonte do corpo de texto: Arial 10, não negrito, alinhamento justificado.

\documentclass[a4paper,10pt]{article}

\usepackage{cmap}		
%\usepackage[utf8]{inputenc}			
\usepackage[english,brazil]{babel}
\usepackage{framed}
\usepackage{hyperref}
\usepackage{amsmath}
\usepackage{graphicx}
\usepackage[colorinlistoftodos]{todonotes}
\usepackage{wrapfig}
\usepackage{lipsum}
\usepackage{listings}
\usepackage{color}
\usepackage{indentfirst}
\usepackage{times}
\usepackage{textcomp}
\usepackage{pgfgantt}
\usepackage{lipsum}

% set document font, font sizes, margin dimensions and spacing
\usepackage{fontspec}
\setmainfont{Arial}
\usepackage[top=15mm,bottom=25mm,left=20mm,right=20mm]{geometry}
\usepackage{setspace}\onehalfspacing
\usepackage{titlesec}
\titleformat*{\section}{\large\bfseries}
\titleformat*{\subsection}{\large\bfseries}
\titleformat*{\subsubsection}{\large\bfseries}
\titleformat*{\paragraph}{\large\bfseries}
\titleformat*{\subparagraph}{\large\bfseries}
\setlength{\parskip}{0.6em}

\newif\ifblackandwhite
\blackandwhitetrue

\usepackage{etoolbox}
\usepackage{longtable}%
\AtBeginEnvironment{longtable}{%
  \addfontfeature{RawFeature=+tnum;-onum}%  <--- requires LuaTeX
}

\usepackage{pdflscape}
%\usepackage[svgnames]{xcolor}
 \usepackage{colortbl}%
   \newcommand{\myrowcolour}{\rowcolor[gray]{0.925}}
\usepackage{booktabs}

\ifblackandwhite
  \newcommand{\cheading}[2]{\textbf{#1\hfill #2}}
  \newcommand{\highest}[1]{\textbf{#1}}% == highest score for question
\else
  \newcommand{\cheading}[2]{\textcolor{Maroon}{\textbf{#1\hfill #2}}}
  \newcommand{\highest}[1]{\textcolor{Maroon}{\textbf{#1}}}%
\fi

\definecolor{mygray}{rgb}{0.4,0.4,0.4}
\definecolor{mygreen}{rgb}{0,0.8,0.6}
\definecolor{myorange}{rgb}{1.0,0.4,0}

\lstdefinestyle{customc}{
  belowcaptionskip=1\baselineskip,
  breaklines=true,
  frame=L,
  xleftmargin=\parindent,
  language=C,
  showstringspaces=false,
  basicstyle=\footnotesize\ttfamily,
  keywordstyle=\bfseries\color{green!40!black},
  commentstyle=\itshape\color{purple!40!black},
  identifierstyle=\color{blue},
  stringstyle=\color{orange},
  numbers=left,
  numbersep=12pt,
  numberstyle=\small\color{mygray},
}
\lstset{escapechar=@,style=customc}

\newcommand{\HRule}{\rule{\linewidth}{0.5mm}}

%Definindo um comando todoin que aceita quebra de linha e fórmulas
\newcommand\todoin[2][]{\todo[inline, caption={2do}, #1]{
\begin{minipage}{\textwidth-4pt}#2\end{minipage}}}

\newcommand\todogeg[2][]{\todo[inline, caption={#2}, color=yellow!100, #1]{
\begin{minipage}{\textwidth-4pt}#2\end{minipage}}}

\newcommand\todovwcm[2][]{\todo[inline, caption={#2}, color=red!100, #1]{
\begin{minipage}{\textwidth-4pt}#2\end{minipage}}}

\begin{document}

\begin{titlepage}
\begin{center}

% logo
\includegraphics[width=0.25\textwidth]{img/unifei_logo_color.png}~\\[1cm]

\textsc{\Large Cenário Atual do Desenvolvimento de Plataformas de E-commerce: Estratégias Técnicas e Impacto na Receita das Empresas}\\[5cm]


% identificacao do edital
\HRule \\[0.4cm]
{\Large \bfseries Revisão bibliográfica e plano de trabalho enviado \\
para análise e deliberação da banca de TFG }
\HRule \\[3cm]

% identificacao do aluno
\large\textbf{Aluno:}\\
Danilo Godofredo de Mattos – Engenharia de Computação\\
Matrícula: 2018009986\\[1cm]

% identificacao do projeto de pesquisa
\large\textbf{Orientação:}\\
Profa. Bárbara Pimenta Caetano - Orientadora\\[1cm]

% identificacao do coordenador
%\large\textbf{Coorientador:}\\
%<Nome do Coorientador> – <Email do Coorientador>\\
%<Instituto> - Unifei\\[1cm]

\vfill

% Bottom of the page
{\large \today}

\end{center}
\end{titlepage}

\newpage
\tableofcontents
\newpage
\section{Introdução}
\label{sc:introducao}

O comércio eletrônico apresenta crescimento acelerado nas últimas décadas, transformando a interação entre empresas e consumidores. As vendas globais de \textit{e-commerce} evoluíram de US\$ 5,13 trilhões em 2022 para projeções de US\$ 6,56 trilhões em 2025, com expectativa de US\$ 8,09 trilhões até 2028 \cite{shopify_relatorio_2025}. Esse crescimento impõe desafios técnicos às organizações que desenvolvem e mantêm plataformas de \textit{e-commerce}, particularmente em desempenho, escalabilidade, experiência do usuário e conversão de vendas.

Decisões técnicas no desenvolvimento de plataformas de \textit{e-commerce} impactam diretamente sucesso e receita das empresas. Desde escolhas arquiteturais padrões de design, distribuição de carga, escalabilidade  até implementações de interface, inteligência artificial e otimizações de performance, cada decisão possui implicações mensuráveis no comportamento do usuário e resultados financeiros.

A diversificação dos modelos de monetização, incluindo pagamento único e assinaturas recorrentes, impõe desafios relacionados à maximização de receita recorrente (\textit{MRR}), aumento do valor do tempo de vida do cliente (\textit{CLTV}) e redução de \textit{churn} (taxa de cancelamento). Estratégias técnicas de \textit{upsell} (venda de produtos/planos superiores), ofertas de retenção no cancelamento e personalização por perfil de assinante representam oportunidades relevantes, mas permanecem pouco exploradas na literatura.

Este trabalho investiga o estado da arte no desenvolvimento de plataformas de \textit{e-commerce}, focando em três dimensões: (i) estratégias arquiteturais e de performance para escalabilidade; (ii) abordagens de \textit{UX/UI} para maximização de conversões; e (iii) integração de inteligência artificial para personalização e automação.

A relevância deste estudo está na necessidade de compreender a relação entre decisões técnicas e impacto financeiro. A literatura apresenta estudos isolados sobre performance arquitetural, otimização \textit{mobile} e sistemas de recomendação, mas há espaço para análise integrada dessas dimensões e sua relação com métricas de receita.





\newpage
\section{Motivação}
\label{sc:motivacao}


Este trabalho é motivado pela lacuna entre conhecimento técnico disponível e sua aplicação estratégica com foco no impacto financeiro de plataformas de \textit{e-commerce}. Três aspectos justificam este estudo.

\textbf{Impacto econômico de decisões técnicas.} O desenvolvimento de plataformas de \textit{e-commerce} envolve investimentos em infraestrutura, arquitetura de software e recursos humanos. Decisões técnicas inadequadas resultam em perdas de receita mensuráveis. Casos documentados demonstram impacto direto: a implementação de \textit{Progressive Web App} no \textit{Alibaba.com} resultou em aumento de 76\% nas conversões \cite{noauthor_alibaba_nodate}, evidenciando as implicações financeiras de escolhas arquiteturais e de \textit{performance}.

\textbf{Complexidade do ecossistema tecnológico.} O desenvolvimento de \textit{e-commerce} apresenta diversidade de abordagens: padrões arquiteturais, \textit{Progressive Web Apps} versus aplicações tradicionais, sistemas de recomendação por filtragem colaborativa versus \textit{deep learning}. Cada decisão envolve \textit{trade-offs} (equilibrar ganhos e perdas) em custo, complexidade, manutenibilidade e performance. Empresas necessitam orientação baseada em evidências para tomar decisões que maximizem retorno sobre investimento.

\textbf{Lacuna entre pesquisa e prática.} A literatura oferece estudos sobre performance arquitetural \cite{ubur_reviewing_2023, zhao_systematic_2024}, otimizações de \textit{UX/UI} \cite{noauthor_pdf_nodate, noauthor_pdf_2025-1} e algoritmos de recomendação \cite{nguyen_personalized_2024}, mas carece de trabalhos que integrem essas dimensões e analisem impacto conjunto na receita. Casos como o \textit{Alibaba.com}, com aumento de 76\% nas conversões após implementação de \textit{PWA} \cite{noauthor_alibaba_nodate}, demonstram potencial de inovações técnicas, mas faltam estudos que sistematizem experiências e identifiquem padrões replicáveis.

Este trabalho busca endereçar essas lacunas, conectando fundamentos técnicos a resultados de negócio. Adicionalmente, insere-se no contexto de formação em Engenharia de Computação da UNIFEI, integrando conhecimentos técnicos com compreensão de seu potencial impacto organizacional e econômico.

\newpage
\section{Revisão bibliográfica}
\label{sc:revisao}





Esta seção apresenta uma revisão crítica da literatura relacionada ao cenário atual do desenvolvimento de plataformas de \textit{e-commerce}, com foco particular nas estratégias técnicas empregadas e seu impacto mensurável na receita e sucesso das empresas. A revisão está organizada em quatro eixos temáticos: (i) arquiteturas de software e desempenho, (ii) experiência do usuário e otimizações de interface, (iii) inteligência artificial e personalização, e (iv) métricas de conversão e análise de tráfego. Cada subseção analisa as estratégias técnicas adotadas, os desafios enfrentados e, quando disponível, o impacto quantitativo dessas decisões na receita e conversão das plataformas.

\subsection{Arquiteturas de Software e Desempenho em \textit{E-commerce}}

A escolha arquitetural de sistemas de \textit{e-commerce} impacta diretamente desempenho e escalabilidade. Ubur \cite{ubur_reviewing_2023} comparou sistemas monolíticos e arquiteturas distribuídas orientadas a eventos, desenvolvendo protótipos e avaliando-os com \textit{Dropwizard Metrics} e \textit{Apache JMeter}. Os resultados mostraram que padrões arquiteturais apresentam desempenho variado conforme contexto: sistemas centralizados oferecem melhores tempos de resposta em cenários de baixa complexidade, enquanto arquiteturas distribuídas performam melhor com alta complexidade e volume de requisições. A escolha depende do contexto técnico e necessidades da plataforma.

Zhao et al. \cite{zhao_systematic_2024} mapearam 109 estudos sobre abordagens arquiteturais para análise de desempenho, identificando que a integração entre arquitetura e performance carece de ferramentas e \textit{benchmarks} para replicação e comparação. Os autores destacam três lacunas: (i) técnicas para domínios emergentes, (ii) consideração de fatores práticos de adoção, e (iii) integração de aprendizado de máquina para análise eficiente.

O uso de servidores distribuídos para \textit{e-commerce} \cite{noauthor_performance_nodate} demonstra como distribuição de carga pode melhorar desempenho em cenários de alta demanda. Essa abordagem oferece resiliência e escalabilidade, características essenciais para crescimento sustentável de receita.

\subsection{Experiência do Usuário e Otimizações de Interface}

\textit{UX} e \textit{UI} desempenham papel fundamental nas taxas de conversão de plataformas de \textit{e-commerce}. A pesquisa sobre impacto de design \textit{UX/UI} nas conversões \cite{noauthor_pdf_nodate} enfatiza que experiência fluida e intuitiva é determinante para sucesso. O estudo identifica que navegação intuitiva, carregamento rápido e design responsivo impactam diretamente satisfação e conclusão de transações.

Complementando essa perspectiva, o estudo sobre o impacto da usabilidade de \textit{websites} e otimização \textit{mobile} na satisfação do cliente e taxas de conversão em negócios de \textit{e-commerce} na Indonésia \cite{noauthor_pdf_2025-1} investigou empiricamente como a usabilidade do site e a otimização para dispositivos móveis afetam a satisfação do cliente e as taxas de conversão de vendas. Os resultados indicam uma correlação positiva significativa entre usabilidade, otimização \textit{mobile} e satisfação do cliente, reforçando a importância de priorizar a experiência em dispositivos móveis, especialmente considerando a crescente predominância do acesso \textit{mobile} no comércio eletrônico.

A otimização de \textit{SEO} (\textit{Search Engine Optimization}) também contribui para a experiência do usuário ao melhorar os tempos de carregamento das páginas. O trabalho sobre como \textit{SEO} torna o carregamento de \textit{websites} mais rápido e auxilia no engajamento de usuários \cite{noauthor_pdf_2025} demonstra que práticas adequadas de \textit{SEO} não apenas melhoram a visibilidade nos mecanismos de busca, mas também resultam em páginas mais leves e rápidas, contribuindo para melhor retenção e engajamento de usuários.

Um exemplo concreto de implementação bem-sucedida é apresentado pelo caso de estudo do \textit{Alibaba.com} \cite{noauthor_alibaba_nodate}, a maior plataforma \textit{B2B} \textit{online} do mundo. Após a implementação de um \textit{Progressive Web App} (\textit{PWA}), a empresa registrou um aumento de 76\% no total de conversões em todos os navegadores. Este caso demonstra o potencial de tecnologias \textit{web} modernas para melhorar simultaneamente a experiência do usuário e as métricas de negócio. A coleção de estudos de caso sobre \textit{PWAs} \cite{noauthor_pwa_nodate} apresenta diversos outros exemplos que corroboram esses benefícios.

\subsection{Inteligência Artificial e Personalização}

A integração de inteligência artificial em plataformas de \textit{e-commerce} tem transformado a interação entre empresas e clientes. Stamkou et al. \cite{stamkou_user_2025} avaliaram a percepção de usuários sobre conteúdo gerado por IA através de questionário aplicado a 223 participantes. Os autores desenvolveram uma loja \textit{online} utilizando \textit{ChatGPT} e \textit{DALL·E} para geração automática de conteúdo, avaliando aspectos de funcionalidade, estética e segurança. Por meio de análise fatorial exploratória (\textit{EFA}), identificaram dois componentes que influenciam a experiência: ``Qualidade de Serviço e Segurança'' e ``\textit{Design} e Estética''. Os participantes avaliaram positivamente o conteúdo gerado por IA, embora demonstrassem cautela quanto à segurança. Os resultados sugerem que IA generativa pode ser ferramenta relevante para desenvolvimento de plataformas de \textit{e-commerce}.

Uma revisão sistemática mais abrangente sobre inteligência artificial em \textit{e-commerce} e \textit{marketing} digital \cite{noauthor_pdf_2025-2} examinou oportunidades, desafios e implicações éticas da aplicação de IA nestes domínios. O estudo enfatiza que, embora a IA ofereça oportunidades significativas para personalização e automação, também apresenta desafios relacionados à privacidade de dados, viés algorítmico e transparência nas decisões automatizadas.

No contexto de sistemas de recomendação, Nguyen et al. \cite{nguyen_personalized_2024} propuseram um modelo de recomendação personalizada baseado em estratégia de recuperação, desenvolvido em colaboração com o Grupo H\&M. O modelo combina filtragem colaborativa, popularidade e \textit{ranking} personalizado bayesiano. Os autores compararam duas técnicas de aprendizado de máquina para avaliação de candidatos: \textit{LightGBM} e Redes Neurais Profundas. O modelo \textit{LightGBM} apresentou desempenho superior, com \textit{MAP@50} de 0,06 contra 0,02 das redes neurais, e \textit{MAR@50} de 0,03 contra 0,01. O \textit{framework} proposto aborda desafios de análise de dados em larga escala e problemas de \textit{cold-start}, contribuindo para melhorar a experiência do usuário em plataformas de \textit{e-commerce}.

\subsection{Métricas de Conversão e Análise de Tráfego}

Métricas de conversão e análise de fontes de tráfego são fundamentais para otimização de plataformas de \textit{e-commerce}. Muralidhar e Lakkanna \cite{muralidhar_clicks_2024} analisaram interações de usuários, métricas de conversão e jornada completa do usuário em uma plataforma de \textit{e-commerce}. O trabalho examinou taxas de saída e sessões em diferentes dispositivos e navegadores, taxas de conversão por fonte de tráfego, e a jornada desde visualização do produto até o \textit{checkout}.

Os resultados indicam necessidade de melhorias em otimização \textit{mobile} e compatibilidade entre navegadores, dado que dispositivos móveis apresentaram taxas de saída mais altas. A análise por fonte de tráfego revelou efetividade variável: mídias anunciadas demonstraram maior potencial, enquanto tráfego de referência e afiliados apresentaram desempenho inferior. O exame da jornada do usuário identificou gargalos no processo de conversão, com lacuna entre interesse e transações completadas. Os autores recomendam melhorias no processo de \textit{checkout} para otimizar conversões.

Em modelos de assinatura, estratégias de \textit{upsell} e retenção são relevantes para maximização de receita recorrente e redução de \textit{churn}. A literatura apresenta análises sobre conversão em modelos transacionais, mas carece de estudos sistemáticos sobre implementação técnica de estratégias como ofertas de \textit{upsell} na jornada de compra, ofertas de retenção no fluxo de cancelamento (\textit{cancellation offers}) e \textit{timing} de apresentação. Em plataformas híbridas, a análise de métricas como \textit{MRR}, \textit{CLTV}, taxa de \textit{churn} e efetividade de estratégias de retenção permanece como área relevante para investigação.

\subsection{Síntese: Estratégias Técnicas e Impacto na Receita}

A literatura revisada demonstra que o cenário atual do desenvolvimento de plataformas de \textit{e-commerce} é caracterizado pela convergência de múltiplas estratégias técnicas que impactam diretamente a receita e o sucesso empresarial. Esta síntese organiza os achados principais relacionando estratégias técnicas com seu impacto mensurável.

\textbf{Decisões arquiteturais e escalabilidade.} As decisões arquiteturais têm impacto direto na capacidade de plataformas de \textit{e-commerce} suportarem crescimento e escalabilidade. O estudo de Ubur (2023) demonstrou que diferentes abordagens arquiteturais apresentam vantagens contextuais: sistemas mais centralizados oferecem benefícios em cenários de baixa complexidade, enquanto arquiteturas distribuídas demonstram melhor desempenho em contextos de alta demanda. Esta constatação evidencia que não existe uma solução arquitetural universalmente superior; a escolha adequada depende do contexto específico do negócio, volume esperado de transações, padrões de crescimento e recursos disponíveis. A escolha arquitetural, portanto, possui implicações diretas na capacidade de uma plataforma suportar crescimento de receita sem degradação de performance.

\textbf{Experiência do usuário como fator crítico de conversão.} A experiência do usuário, especialmente em dispositivos móveis, é consistentemente identificada como fator crítico para taxas de conversão e, consequentemente, para a receita. O caso do \textit{Alibaba.com} exemplifica este impacto de forma quantitativa: a implementação de \textit{PWA} resultou em aumento de 76\% nas conversões totais,  tradução direta de decisão técnica em crescimento de receita. A literatura revisada indica correlação positiva significativa entre usabilidade, otimização \textit{mobile} e satisfação do cliente, com impacto mensurável nas taxas de conversão de vendas.

\textbf{Personalização via IA: oportunidades e desafios.} A integração de inteligência artificial oferece oportunidades substanciais para personalização e automação, com sistemas de recomendação demonstrando eficácia mensurável. O \textit{framework} desenvolvido por Nguyen et al. (2024) para o Grupo H\&M ilustra como técnicas de aprendizado de máquina podem contribuir para melhorar a experiência do usuário e impulsionar vendas. No entanto, a literatura também evidencia desafios relacionados à aceitação dos usuários, especialmente em questões de segurança e privacidade, que podem impactar negativamente a confiança e, consequentemente, as taxas de conversão.

\textbf{Otimização de métricas como estratégia contínua.} A análise de Muralidhar e Lakkanna (2024) sobre a jornada do usuário e métricas de conversão destaca que o desenvolvimento de plataformas de \textit{e-commerce} não é um processo estático, mas requer otimização contínua baseada em dados. A identificação de gargalos no processo de \textit{checkout}, variações nas taxas de conversão por fonte de tráfego e diferenças de performance entre dispositivos e navegadores representam oportunidades concretas para incremento de receita através de melhorias técnicas direcionadas.

\subsubsection{Oportunidades de Pesquisa}

Apesar dos avanços documentados, ainda há oportunidades de pesquisa sobre estratégias técnicas e seu impacto na receita de plataformas de \textit{e-commerce}:

\begin{enumerate}
    \item \textbf{\textit{Frameworks} integrados de análise:} A literatura apresenta estudos isolados sobre aspectos específicos de arquitetura \cite{ubur_reviewing_2023, zhao_systematic_2024}, \textit{UX} \cite{noauthor_pdf_nodate, noauthor_pdf_2025-1} e IA \cite{stamkou_user_2025, nguyen_personalized_2024}, mas carece de \textit{frameworks} integrados que combinem análise arquitetural, desempenho, experiência do usuário e métricas de negócio de forma holística. Esta lacuna dificulta a tomada de decisões estratégicas que considere \textit{trade-offs} entre diferentes dimensões técnicas.

    \item \textbf{Estudos de longo prazo sobre \textit{ROI} de implementações de IA:} Embora estudos demonstrem a eficácia técnica de sistemas de IA e recomendação \cite{stamkou_user_2025, nguyen_personalized_2024, noauthor_pdf_2025-2}, há carência de estudos de longo prazo que avaliem o retorno sobre investimento dessas implementações, considerando custos de desenvolvimento, manutenção e impacto em métricas de satisfação, fidelização e \textit{lifetime value} de clientes.

    \item \textbf{\textit{Benchmarks} padronizados para comparação arquitetural:} A ausência de \textit{benchmarks} padronizados e \textit{datasets} de referência para avaliação de diferentes abordagens arquiteturais em contextos de \textit{e-commerce}, conforme identificado por Zhao et al. \cite{zhao_systematic_2024}, dificulta comparações objetivas e decisões baseadas em evidências sobre escolhas arquiteturais.

    \item \textbf{Otimização integrada via \textit{ML/AI}:} A exploração de técnicas modernas de aprendizado de máquina para otimização integrada e automática de múltiplos aspectos de plataformas de \textit{e-commerce} (arquitetura, performance, \textit{UX}, personalização) permanece limitada, representando oportunidade significativa para pesquisa futura.

    \item \textbf{Relação quantitativa entre decisões técnicas e receita:} Embora casos como o \textit{Alibaba.com} \cite{noauthor_alibaba_nodate} forneçam dados quantitativos sobre impacto de implementações específicas, faltam estudos sistemáticos que estabeleçam relações causais e quantitativas entre decisões técnicas específicas (por exemplo, tempo de resposta de \textit{APIs}, qualidade de recomendações, design de \textit{checkout}) e métricas de receita, considerando diferentes contextos de negócio e segmentos de mercado.

    \item \textbf{Estratégias de monetização via assinatura:} Há carência de estudos que investiguem sistematicamente a implementação técnica de estratégias de \textit{upsell}, retenção e ofertas de cancelamento em modelos de \textit{e-commerce} baseados em assinatura. Pesquisas futuras devem explorar o impacto de diferentes abordagens de \textit{timing}, \textit{messaging} e tipo de oferta em métricas críticas como \textit{MRR}, \textit{CLTV}, taxa de \textit{churn} e receita incremental, especialmente em plataformas híbridas que combinam pagamento único e assinaturas recorrentes.
\end{enumerate}

\newpage
\section{Objetivos específicos}
\label{sc:objetivos}


O objetivo geral deste trabalho é analisar o cenário atual do desenvolvimento de plataformas de \textit{e-commerce}, identificando e caracterizando as principais estratégias técnicas empregadas e avaliando seu impacto na receita e no sucesso das empresas. Para alcançar este objetivo geral, foram definidos os seguintes objetivos específicos:

\begin{enumerate}
    \item \textbf{Mapear as estratégias arquiteturais}: Identificar e caracterizar as principais abordagens arquiteturais empregadas no desenvolvimento de plataformas de \textit{e-commerce} modernas, analisando os \textit{trade-offs} entre diferentes padrões de design, estratégias de distribuição e modelos de escalabilidade em termos de desempenho, custo de desenvolvimento e capacidade de adaptação a diferentes contextos de negócio.

    \item \textbf{Analisar estratégias de otimização de experiência do usuário}: Examinar as principais técnicas e tecnologias utilizadas para otimização da experiência do usuário em plataformas de \textit{e-commerce}, com ênfase em otimizações \textit{mobile}, \textit{Progressive Web Apps} (\textit{PWAs}), design responsivo e otimizações de performance, relacionando essas estratégias com métricas de conversão e satisfação do cliente.

    \item \textbf{Avaliar o papel da inteligência artificial}: Investigar a aplicação de técnicas de inteligência artificial e aprendizado de máquina em plataformas de \textit{e-commerce}, incluindo sistemas de recomendação, personalização de conteúdo e geração de conteúdo via IA generativa, analisando a eficácia dessas implementações e os desafios relacionados à aceitação dos usuários, privacidade e segurança.

    \item \textbf{Correlacionar decisões técnicas com impacto na receita}: Identificar e analisar casos documentados que demonstrem relação mensurável entre decisões técnicas específicas (arquitetura, \textit{UX/UI}, IA, estratégias de marketing) e métricas de negócio (taxas de conversão, receita, satisfação do cliente, retenção), estabelecendo quando possível relações quantitativas entre estratégias técnicas e resultados financeiros, considerando tanto modelos de pagamento único quanto assinaturas recorrentes.

    \item \textbf{Identificar lacunas e oportunidades}: Mapear lacunas no conhecimento atual sobre desenvolvimento de plataformas de \textit{e-commerce} e identificar oportunidades para pesquisa futura, contribuindo para o avanço do campo e para orientação de decisões estratégicas em organizações que desenvolvem ou operam plataformas de comércio eletrônico.
\end{enumerate}



\newpage
\section{Aplicação e observação em aplicações de e-commerce}
\label{sc:aplicacao_e_observacao}

\subsection{Análise da aplicação das estratégias técnicas em plataformas de e-commerce - Plataforma Achievece}

 Consiste em uma observação das estratégias técnicas identificadas na revisão bibliográfica através de estudo da plataforma Achievece, um e-commerce americano especializado em venda de cursos online. O estudo busca verificar o impacto nas métricas de receita e conversão da empresa.

\subsubsection{Contexto da Plataforma}

Achievece é uma plataforma de e-commerce de cursos online que apresenta características representativas no cenário de educação continuada para médicos nos Estados Unidos, com volume médio de tráfego de aproximadamente 30 mil visitantes mensais e receita diária na ordem de US\$ 2.500. A plataforma opera com modelo de negócio híbrido, oferecendo tanto cursos com pagamento único quanto planos de assinatura recorrente, permitindo análise de estratégias específicas para ambos os modelos de monetização. A escolha desta plataforma como objeto de estudo justifica-se por: (i) acesso autorizado aos dados de analytics e métricas de negócio, respeitando informações sigilosas da empresa, (ii) possibilidade de implementação controlada de modificações técnicas, (iii) volume de tráfego suficiente para análise estatística significativa, (iv) representatividade do segmento de e-learning no mercado americano, e (v) diversidade de modelos de monetização que permite análise comparativa.

\subsubsection{Possíveis pontos de observação }

\textbf{Item 1: Testes A/B de Interface do Usuário}

Implementação de testes A/B controlados utilizando a plataforma GrowthBook para avaliar o impacto de modificações de UI/UX em taxas de conversão. Variações planejadas incluem:
\begin{itemize}
    \item Redesign do processo de checkout (redução de etapas, otimização de formulários)
    \item Modificações na página de detalhes do produto (disposição de informações, call-to-action)
    \item Variações de design responsivo para dispositivos móveis
    \item Testes de copy e messaging (títulos, descrições, CTAs)
    \item Elementos de urgência e escassez nas páginas de produto
\end{itemize}

\textbf{Avaliação}: Divisão aleatória e controlada do tráfego entre versões A (controle) e B (variação) através do GrowthBook, com período mínimo de 2 semanas por experimento para garantir significância estatística considerando o volume de tráfego da plataforma. Métricas monitoradas: taxa de conversão, taxa de abandono de carrinho, tempo médio na página, receita por visitante.

\textbf{Item 2: Avaliação do uso de Progressive Web App (PWA)}

Análise do impacto da implementação de funcionalidades de Progressive Web App na plataforma, seguindo evidências documentadas em casos como o Alibaba.com. Funcionalidades a serem avaliadas:
\begin{itemize}
    \item Redução de tempo de carregamento através de estratégias de cache
    \item Funcionalidade offline para navegação no catálogo de cursos
    \item Implementação de notificações push para engajamento de usuários
    \item Capacidade de instalação no dispositivo (add to home screen)
\end{itemize}

\textbf{Avaliação}: Comparação de métricas antes e depois da implementação das funcionalidades PWA, com período de baseline de 4 semanas pré-implementação e 8 semanas pós-implementação. Análise de impacto em: tempo de carregamento, taxa de rejeição, sessões por usuário, conversões totais, receita. A análise considerará possíveis efeitos de sazonalidade através de comparação com períodos equivalentes do ano anterior, quando disponível.

\textbf{Item 3: Otimizações de Performance e Arquitetura}

Implementação incremental de otimizações técnicas baseadas nas estratégias identificadas na revisão bibliográfica:
\begin{itemize}
    \item Otimização de imagens e assets (lazy loading, compressão, formatos modernos)
    \item Implementação de CDN para distribuição de conteúdo estático
    \item Otimização de queries e cache de dados
    \item Análise de distribuição de carga e identificação de gargalos
\end{itemize}

\textbf{Avaliação}: Monitoramento contínuo de métricas de performance (Core Web Vitals: LCP, FID, CLS) através do Google Lighthouse e correlação com métricas de negócio (taxa de conversão, receita por sessão). Cada otimização será implementada de forma isolada quando possível, permitindo identificar contribuições individuais para melhoria de performance.

\textbf{Item 4: Personalização e Recomendação}

Implementação de sistema básico de recomendação de cursos baseado em:
\begin{itemize}
    \item Filtragem colaborativa (cursos visualizados por usuários similares)
    \item Análise de navegação e comportamento
    \item Categorização e tags de conteúdo
\end{itemize}

\textbf{Avaliação}: Teste A/B através do GrowthBook comparando versão com e sem recomendações personalizadas. Métricas: taxa de cliques em recomendações, cross-selling, ticket médio, receita por usuário.

\textbf{Item 5: Análise de SEO e Impacto em Tráfego Orgânico}

Implementação de otimizações de SEO técnico e de conteúdo para avaliar impacto em visibilidade e conversões:
\begin{itemize}
    \item Otimização de meta tags, títulos e descrições
    \item Melhoria de estrutura de URLs e breadcrumbs
    \item Implementação de schema markup para rich snippets
    \item Otimização de velocidade de carregamento (já contemplada no Ponto 3)
    \item Criação e otimização de conteúdo relevante
\end{itemize}

\textbf{Avaliação}: Monitoramento de posicionamento em mecanismos de busca (Google Search Console), volume de tráfego orgânico e correlação com receita. Análise do ROI de tráfego orgânico comparado a outras fontes.

\textbf{Item 6: Análise Comparativa de Fontes de Tráfego}

Análise detalhada do comportamento e performance de diferentes fontes de tráfego para validar os achados de Muralidhar e Lakkanna (2024):
\begin{itemize}
    \item Tráfego orgânico (busca)
    \item Tráfego pago (Google Ads, Meta Ads)
    \item Tráfego direto
    \item Tráfego de referência
    \item Tráfego social (redes sociais)
\end{itemize}

\textbf{Avaliação}: Segmentação de análise por fonte de tráfego, comparando: taxa de conversão, valor médio do pedido, receita por visitante, custo de aquisição (quando aplicável), ROI. Identificação de fontes com maior potencial de otimização e melhor retorno sobre investimento.

\textbf{Item 7: Testes de Recuperação de Carrinho Abandonado}

Implementação e teste A/B de estratégias de recuperação de carrinhos abandonados:
\begin{itemize}
    \item Emails automatizados de lembrete de carrinho abandonado
    \item Variações de timing de envio (1h, 24h, 48h após abandono)
    \item Testes de copy e ofertas (desconto vs urgência vs valor agregado)
    \item Análise de push notifications via PWA para recuperação
\end{itemize}

\textbf{Avaliação}: Teste A/B controlado através do GrowthBook para diferentes abordagens de messaging. Métricas: taxa de abertura de emails, taxa de cliques, taxa de recuperação de carrinho, receita recuperada.

\textbf{Item 8: Estratégias de Upsell e Retenção para Assinaturas}

Implementação e análise de estratégias de marketing técnicas específicas para o modelo de assinatura de cursos, focando em maximização de receita recorrente (MRR) e redução de churn:
\begin{itemize}
    \item Upsell durante a jornada de compra: ofertas de upgrade de plano no momento da compra inicial
    \item Upsell pós-compra: ofertas de cursos complementares ou upgrades de plano para assinantes ativos
    \item Cancellation offers: ofertas de retenção apresentadas no fluxo de cancelamento (descontos, pausa de assinatura, downgrade de plano)
    \item Timing de ofertas: teste de diferentes momentos para apresentação de upsells (checkout, onboarding, uso ativo)
\end{itemize}

\textbf{Avaliação}: Implementação de testes A/B através do GrowthBook para diferentes estratégias de upsell e ofertas de cancelamento. Grupo controle: sem ofertas de upsell/retenção. Grupos de variação: diferentes tipos de ofertas, timings e messaging. Análise de fluxo de cancelamento com identificação de pontos de abandono através do Lucky Orange.

\textbf{Métricas específicas para assinaturas}:
\begin{itemize}
    \item Taxa de conversão de upsell (percentual de usuários que aceitam ofertas)
    \item Receita incremental por upsell
    \item Customer Lifetime Value (CLTV) médio por segmento
    \item Taxa de churn (cancelamentos mensais)
    \item Taxa de retenção via cancellation offers
    \item Monthly Recurring Revenue (MRR) e impacto de cada estratégia
    \item Average Revenue Per User (ARPU)
    \item Tempo médio de assinatura antes do cancelamento
\end{itemize}

A análise considerará a segmentação de usuários por tipo de plano (básico, intermediário, premium), tempo de assinatura e padrões de uso para identificar quais estratégias são mais efetivas para cada perfil de cliente.

\subsubsection{Coleta e Análise de Dados}

\textbf{Ferramentas de instrumentação}:
\begin{itemize}
    \item Google Analytics 4 para análise de comportamento e conversões
    \item Google Lighthouse para métricas de performance e Core Web Vitals
    \item Google Search Console para dados de tráfego orgânico e SEO
    \item GrowthBook para execução e análise de testes A/B
    \item Lucky Orange para análise comportamental (heatmaps, session recordings, funnels)
    \item Ferramentas próprias de analytics da plataforma para dados de receita
\end{itemize}

\textbf{Métricas primárias}: Taxa de conversão, receita total, receita por visitante (RPV), valor médio do pedido (AOV), Monthly Recurring Revenue (MRR), Customer Lifetime Value (CLTV).

\textbf{Métricas secundárias}: Taxa de rejeição, tempo médio na página, páginas por sessão, taxa de abandono de carrinho, tempo de carregamento, Core Web Vitals (LCP, FID, CLS), taxa de recuperação de carrinho, sessões por usuário, taxa de churn, taxa de retenção, ARPU.

\textbf{Métricas específicas de assinaturas}: Taxa de conversão de upsell, taxa de retenção via cancellation offers, tempo médio de assinatura, taxa de upgrade/downgrade entre planos, receita incremental por estratégia de upsell.

\textbf{Análise estatística}: Testes de significância estatística (teste t para médias, teste qui-quadrado para proporções) com nível de confiança de 95\% para validação de hipóteses. Cálculo de tamanho de amostra considerando o volume de tráfego da plataforma (aproximadamente 30.000 visitantes mensais) para garantir poder estatístico adequado. Análise de séries temporais para identificação de tendências e sazonalidades. Correção de Bonferroni quando aplicável para testes múltiplos.

\subsubsection{Considerações Éticas e Limitações}

O estudo será conduzido respeitando a privacidade dos usuários, utilizando apenas dados agregados e anonimizados em conformidade com a Lei Geral de Proteção de Dados (LGPD). Todas as variações de experimentos A/B manterão funcionalidades essenciais, garantindo que nenhum grupo de usuários tenha experiência degradada. Informações sigilosas da empresa serão preservadas, com apresentação de dados de forma agregada ou normalizada quando necessário.

Limitações reconhecidas incluem: (i) generalização dos resultados limitada ao contexto de e-commerce de cursos online nos Estados Unidos, (ii) período de observação de quatro meses, que pode não capturar completamente efeitos de longo prazo ou variações anuais, (iii) impossibilidade de isolar completamente fatores externos como sazonalidade do mercado educacional, campanhas de marketing não controladas pelo estudo, mudanças na concorrência e variações macroeconômicas, (iv) volume de tráfego que, embora significativo, pode limitar a detecção de efeitos pequenos em certos pontos de observação.

\newpage
\section{Resultados esperados}
\label{sc:resultados}

Espera-se observar os seguintes resultados:

\subsection{Impacto de Otimizações de Interface e \textit{UX}}

\begin{itemize}
    \item Dados quantitativos sobre o impacto de \textit{redesign} do processo de \textit{checkout} na taxa de conversão
    \item Identificação de variações de design responsivo \textit{mobile} com melhor desempenho em métricas de conversão
    \item \textit{Insights} comportamentais através de \textit{heatmaps} e \textit{session recordings} sobre pontos de fricção na jornada do usuário
    \item Quantificação do impacto de elementos de urgência e \textit{copy} otimizado na receita
\end{itemize}

\subsection{Efeitos da Implementação de \textit{PWA} e Otimizações de Performance}

\begin{itemize}
    \item Melhoria em \textit{Core Web Vitals} (\textit{LCP}, \textit{FID}, \textit{CLS}) após implementação de \textit{PWA}
    \item Correlação entre melhorias de performance técnica e métricas de negócio (taxa de conversão, receita, \textit{bounce rate})
    \item Dados sobre impacto de notificações \textit{push} na retenção e engajamento de usuários
\end{itemize}

\subsection{Efetividade de Personalização e \textit{SEO}}

\begin{itemize}
    \item Dados de eficácia de sistema de recomendação personalizada na conversão e \textit{AOV}
    \item Impacto de otimizações de \textit{SEO} técnico no tráfego orgânico e posicionamento em mecanismos de busca
    \item Análise comparativa de performance e \textit{ROI} por fonte de tráfego (orgânico, pago, direto, social, referência)
    \item Taxa de recuperação e receita incremental de estratégias de recuperação de carrinho abandonado
\end{itemize}

\subsection{Estratégias de \textit{Upsell} e Retenção em Modelo de Assinatura}

\begin{itemize}
    \item Taxa de conversão de ofertas de \textit{upsell} apresentadas durante o \textit{checkout} vs. pós-compra
    \item Efetividade de diferentes tipos de \textit{cancellation offers} (descontos, pausa de assinatura, \textit{downgrade}) na redução de \textit{churn}
    \item Impacto quantitativo de estratégias de \textit{upsell} e retenção em métricas críticas de assinatura:
    \begin{itemize}
        \item \textit{Monthly Recurring Revenue} (\textit{MRR}) incremental
        \item \textit{Customer Lifetime Value} (\textit{CLTV}) médio por coorte
        \item Taxa de \textit{churn} antes e depois da implementação de ofertas de retenção
        \item \textit{Average Revenue Per User} (\textit{ARPU}) por tipo de plano
        \item Taxa de \textit{upgrade} e \textit{downgrade} entre planos
    \end{itemize}
    \item Identificação de perfis de clientes (segmentos) mais responsivos a cada tipo de estratégia de \textit{upsell} e retenção
    \item \textit{Timing} ótimo para apresentação de ofertas (momento da jornada com maior taxa de conversão)
\end{itemize}

\subsection{Análise Integrada e Correlações}

\begin{itemize}
    \item Identificação de padrões entre experimentos e possíveis sinergias entre diferentes estratégias técnicas
    \item Síntese de aprendizados práticos para priorização de estratégias técnicas em plataformas de \textit{e-commerce}
\end{itemize}



\newpage
\section{Cronograma de atividades do TFG}
\label{sc:cronograma}

\begin{longtable}{p{\textwidth}}
% pairs: absolute number (percentage)
\toprule%
\myrowcolour%
\bfseries F.1: Planejamento e Configuração (Agosto - Setembro) \\
\midrule
Planejamento detalhado dos experimentos a serem conduzidos na plataforma Achievece e configuração das ferramentas de instrumentação e analytics.
\\
\midrule
\myrowcolour%
\bfseries Atividades planejadas para a fase 1 \\
\midrule
A.1.1 Planejamento detalhado dos experimentos \\
A.1.2 Configuração das ferramentas de instrumentação (Google Analytics, GrowthBook, Lucky Orange) \\
A.1.3 Estabelecimento de baseline de métricas \\
A.1.4 Definição de critérios de significância estatística \\
\midrule
\myrowcolour%
\bfseries Resultados esperados para a fase 1\\
\midrule
Plano experimental detalhado aprovado. Ferramentas de analytics configuradas e funcionais. Baseline de métricas estabelecido para comparações futuras. Critérios de análise estatística definidos.\\
\toprule%
\myrowcolour%
\bfseries F.2: Experimentos de Interface e UX (Setembro - Outubro) \\
\midrule
Implementação e execução de testes A/B focados em otimizações de interface do usuário, processo de checkout e elementos de conversão.
\\
\midrule
\myrowcolour%
\bfseries Atividades planejadas para a fase 2 \\
\midrule
A.2.1 Teste A/B: Redesign do processo de checkout \\
A.2.2 Teste A/B: Modificações na página de detalhes do produto \\
A.2.3 Teste A/B: Variações de design responsivo mobile \\
A.2.4 Teste A/B: Copy, messaging e elementos de urgência \\
A.2.5 Análise comportamental com Lucky Orange (heatmaps, recordings) \\
A.2.6 Coleta e análise de dados de conversão \\
\midrule
\myrowcolour%
\bfseries Resultados esperados para a fase 2 \\
\midrule
Identificação de variações de UI/UX com melhor desempenho. Dados quantitativos sobre impacto de mudanças de interface em taxa de conversão e receita. Insights comportamentais sobre navegação e pontos de fricção.\\
\toprule%
\myrowcolour%
\bfseries F.3: Implementação PWA e Otimizações de Performance (Outubro - Novembro) \\
\midrule
Implementação de funcionalidades Progressive Web App e otimizações técnicas de performance, com análise de impacto em métricas de negócio.
\\
\midrule
\myrowcolour%
\bfseries Atividades planejadas para a fase 3 \\
\midrule
A.3.1 Implementação de service workers e estratégias de cache \\
A.3.2 Configuração de funcionalidades offline e notificações push \\
A.3.3 Otimização de imagens e assets (lazy loading, compressão) \\
A.3.4 Implementação de CDN para conteúdo estático \\
A.3.5 Otimização de queries e cache de dados \\
A.3.6 Monitoramento de Core Web Vitals e correlação com conversão \\
A.3.7 Análise comparativa antes/depois da implementação PWA \\
\midrule
\myrowcolour%
\bfseries Resultados esperados para a fase 3 \\
\midrule
PWA funcional com cache estratégico e notificações push. Melhoria mensurável em Core Web Vitals (LCP, FID, CLS). Dados sobre impacto de performance em taxa de conversão e receita. Comparação quantitativa com baseline pré-implementação.\\
\toprule%
\myrowcolour%
\bfseries F.4: Personalização, SEO e Estratégias de Assinatura (Novembro - Dezembro) \\
\midrule
Implementação de sistema de recomendação, otimizações de SEO, análise comparativa detalhada de fontes de tráfego, estratégias de recuperação de carrinho e implementação de estratégias de upsell e retenção para modelo de assinatura.
\\
\midrule
\myrowcolour%
\bfseries Atividades planejadas para a fase 4 \\
\midrule
A.4.1 Implementação de sistema básico de recomendação de cursos \\
A.4.2 Teste A/B: versão com vs sem recomendações personalizadas \\
A.4.3 Otimizações de SEO técnico (meta tags, schema markup, URLs) \\
A.4.4 Monitoramento de posicionamento e tráfego orgânico \\
A.4.5 Análise comparativa de fontes de tráfego (orgânico, pago, direto, social) \\
A.4.6 Teste A/B: estratégias de recuperação de carrinho abandonado \\
A.4.7 Análise de ROI por fonte de tráfego \\
A.4.8 Implementação de ofertas de upsell (checkout, pós-compra) \\
A.4.9 Teste A/B: diferentes tipos de cancellation offers \\
A.4.10 Análise de churn e padrões de cancelamento \\
\midrule
\myrowcolour%
\bfseries Resultados esperados para a fase 4 \\
\midrule
Sistema de recomendação funcional com dados de eficácia. Melhoria em posicionamento orgânico e tráfego qualificado. Análise completa de performance por fonte de tráfego. Estratégias validadas de recuperação de carrinho com métricas de taxa de recuperação e receita. Dados sobre efetividade de upsells e cancellation offers com impacto em MRR, CLTV e taxa de churn. Identificação de perfis de clientes mais responsivos a cada estratégia.\\
\toprule%
\myrowcolour%
\bfseries F.5: Análise de Dados e Redação (Dezembro) \\
\midrule
Consolidação de todos os dados coletados, análise estatística integrada e redação dos resultados e conclusões do trabalho.
\\
\midrule
\myrowcolour%
\bfseries Atividades planejadas para a fase 5 \\
\midrule
A.5.1 Consolidação de dados de todos os experimentos \\
A.5.2 Análise estatística integrada (significância, correlações) \\
A.5.3 Identificação de padrões e insights cross-experimentos \\
A.5.4 Redação da seção de resultados \\
A.5.5 Discussão: validação das hipóteses da revisão bibliográfica \\
A.5.6 Redação de conclusões e recomendações práticas \\
A.5.7 Revisão e finalização do documento completo \\
\midrule
\myrowcolour%
\bfseries Resultados esperados para a fase 5 \\
\midrule
Documento completo do TFG com análise integrada de todos os experimentos. Validação ou refutação das hipóteses identificadas na revisão bibliográfica. Recomendações práticas baseadas em evidências para desenvolvimento de plataformas de e-commerce. Identificação de lacunas para pesquisas futuras.\\

\bottomrule
\end{longtable}



\noindent
\makebox[\textwidth][c]{%
\begin{ganttchart}[
y unit title=.8cm,
y unit chart=.7cm,
x unit = 3.5cm,
hgrid,
vgrid,
time slot format=isodate-yearmonth,
time slot unit=month,
inline,
bar height=0.7,
bar/.append style={fill=gray!50, inner sep=0pt},
title height=1,
canvas/.append style={draw=none}
]{2025-08}{2025-12}
\gantttitlecalendar{year, month} \\
  \ganttbar{F.1: Planejamento e Config.}{2025-08}{2025-09} \\
  \ganttbar{F.2: Testes A/B UI/UX}{2025-09}{2025-10} \\
  \ganttbar{F.3: PWA e Performance}{2025-10}{2025-11} \\
  \ganttbar{F.4: Personalização e SEO}{2025-11}{2025-12} \\
  \ganttbar{F.5: Análise e Redação}{2025-12}{2025-12} \\

\end{ganttchart}%
}

\renewcommand\refname{Referências Bibliográficas}
\bibliographystyle{abntex2-alf}
%\bibliographystyle{IEEEtran}
\bibliography{referencias.bib}
%\listoftodos

\end{document}