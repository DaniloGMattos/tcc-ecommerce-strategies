\newpage
\section{Cronograma de atividades do TFG}
\label{sc:cronograma}

\begin{longtable}{p{\textwidth}}
% pairs: absolute number (percentage)
\toprule%
\myrowcolour%
\bfseries F.1: Planejamento e Configuração (Agosto - Setembro) \\
\midrule
Planejamento detalhado dos experimentos a serem conduzidos na plataforma Achievece e configuração das ferramentas de instrumentação e analytics.
\\
\midrule
\myrowcolour%
\bfseries Atividades planejadas para a fase 1 \\
\midrule
A.1.1 Planejamento detalhado dos experimentos \\
A.1.2 Configuração das ferramentas de instrumentação (Google Analytics, GrowthBook, Lucky Orange) \\
A.1.3 Estabelecimento de baseline de métricas \\
A.1.4 Definição de critérios de significância estatística \\
\midrule
\myrowcolour%
\bfseries Resultados esperados para a fase 1\\
\midrule
Plano experimental detalhado aprovado. Ferramentas de analytics configuradas e funcionais. Baseline de métricas estabelecido para comparações futuras. Critérios de análise estatística definidos.\\
\toprule%
\myrowcolour%
\bfseries F.2: Experimentos de Interface e UX (Setembro - Outubro) \\
\midrule
Implementação e execução de testes A/B focados em otimizações de interface do usuário, processo de checkout e elementos de conversão.
\\
\midrule
\myrowcolour%
\bfseries Atividades planejadas para a fase 2 \\
\midrule
A.2.1 Teste A/B: Redesign do processo de checkout \\
A.2.2 Teste A/B: Modificações na página de detalhes do produto \\
A.2.3 Teste A/B: Variações de design responsivo mobile \\
A.2.4 Teste A/B: Copy, messaging e elementos de urgência \\
A.2.5 Análise comportamental com Lucky Orange (heatmaps, recordings) \\
A.2.6 Coleta e análise de dados de conversão \\
\midrule
\myrowcolour%
\bfseries Resultados esperados para a fase 2 \\
\midrule
Identificação de variações de UI/UX com melhor desempenho. Dados quantitativos sobre impacto de mudanças de interface em taxa de conversão e receita. Insights comportamentais sobre navegação e pontos de fricção.\\
\toprule%
\myrowcolour%
\bfseries F.3: Implementação PWA e Otimizações de Performance (Outubro - Novembro) \\
\midrule
Implementação de funcionalidades Progressive Web App e otimizações técnicas de performance, com análise de impacto em métricas de negócio.
\\
\midrule
\myrowcolour%
\bfseries Atividades planejadas para a fase 3 \\
\midrule
A.3.1 Implementação de service workers e estratégias de cache \\
A.3.2 Configuração de funcionalidades offline e notificações push \\
A.3.3 Otimização de imagens e assets (lazy loading, compressão) \\
A.3.4 Implementação de CDN para conteúdo estático \\
A.3.5 Otimização de queries e cache de dados \\
A.3.6 Monitoramento de Core Web Vitals e correlação com conversão \\
A.3.7 Análise comparativa antes/depois da implementação PWA \\
\midrule
\myrowcolour%
\bfseries Resultados esperados para a fase 3 \\
\midrule
PWA funcional com cache estratégico e notificações push. Melhoria mensurável em Core Web Vitals (LCP, FID, CLS). Dados sobre impacto de performance em taxa de conversão e receita. Comparação quantitativa com baseline pré-implementação.\\
\toprule%
\myrowcolour%
\bfseries F.4: Personalização, SEO e Estratégias de Assinatura (Novembro - Dezembro) \\
\midrule
Implementação de sistema de recomendação, otimizações de SEO, análise comparativa detalhada de fontes de tráfego, estratégias de recuperação de carrinho e implementação de estratégias de upsell e retenção para modelo de assinatura.
\\
\midrule
\myrowcolour%
\bfseries Atividades planejadas para a fase 4 \\
\midrule
A.4.1 Implementação de sistema básico de recomendação de cursos \\
A.4.2 Teste A/B: versão com vs sem recomendações personalizadas \\
A.4.3 Otimizações de SEO técnico (meta tags, schema markup, URLs) \\
A.4.4 Monitoramento de posicionamento e tráfego orgânico \\
A.4.5 Análise comparativa de fontes de tráfego (orgânico, pago, direto, social) \\
A.4.6 Teste A/B: estratégias de recuperação de carrinho abandonado \\
A.4.7 Análise de ROI por fonte de tráfego \\
A.4.8 Implementação de ofertas de upsell (checkout, pós-compra) \\
A.4.9 Teste A/B: diferentes tipos de cancellation offers \\
A.4.10 Análise de churn e padrões de cancelamento \\
\midrule
\myrowcolour%
\bfseries Resultados esperados para a fase 4 \\
\midrule
Sistema de recomendação funcional com dados de eficácia. Melhoria em posicionamento orgânico e tráfego qualificado. Análise completa de performance por fonte de tráfego. Estratégias validadas de recuperação de carrinho com métricas de taxa de recuperação e receita. Dados sobre efetividade de upsells e cancellation offers com impacto em MRR, CLTV e taxa de churn. Identificação de perfis de clientes mais responsivos a cada estratégia.\\
\toprule%
\myrowcolour%
\bfseries F.5: Análise de Dados e Redação (Dezembro) \\
\midrule
Consolidação de todos os dados coletados, análise estatística integrada e redação dos resultados e conclusões do trabalho.
\\
\midrule
\myrowcolour%
\bfseries Atividades planejadas para a fase 5 \\
\midrule
A.5.1 Consolidação de dados de todos os experimentos \\
A.5.2 Análise estatística integrada (significância, correlações) \\
A.5.3 Identificação de padrões e insights cross-experimentos \\
A.5.4 Redação da seção de resultados \\
A.5.5 Discussão: validação das hipóteses da revisão bibliográfica \\
A.5.6 Redação de conclusões e recomendações práticas \\
A.5.7 Revisão e finalização do documento completo \\
\midrule
\myrowcolour%
\bfseries Resultados esperados para a fase 5 \\
\midrule
Documento completo do TFG com análise integrada de todos os experimentos. Validação ou refutação das hipóteses identificadas na revisão bibliográfica. Recomendações práticas baseadas em evidências para desenvolvimento de plataformas de e-commerce. Identificação de lacunas para pesquisas futuras.\\

\bottomrule
\end{longtable}



\noindent
\makebox[\textwidth][c]{%
\begin{ganttchart}[
y unit title=.8cm,
y unit chart=.7cm,
x unit = 3.5cm,
hgrid,
vgrid,
time slot format=isodate-yearmonth,
time slot unit=month,
inline,
bar height=0.7,
bar/.append style={fill=gray!50, inner sep=0pt},
title height=1,
canvas/.append style={draw=none}
]{2025-08}{2025-12}
\gantttitlecalendar{year, month} \\
  \ganttbar{F.1: Planejamento e Config.}{2025-08}{2025-09} \\
  \ganttbar{F.2: Testes A/B UI/UX}{2025-09}{2025-10} \\
  \ganttbar{F.3: PWA e Performance}{2025-10}{2025-11} \\
  \ganttbar{F.4: Personalização e SEO}{2025-11}{2025-12} \\
  \ganttbar{F.5: Análise e Redação}{2025-12}{2025-12} \\

\end{ganttchart}%
}
