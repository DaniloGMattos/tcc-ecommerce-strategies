\newpage
\section{Introdução}
\label{sc:introducao}

O comércio eletrônico apresenta crescimento acelerado nas últimas décadas, transformando a interação entre empresas e consumidores. As vendas globais de \textit{e-commerce} evoluíram de US\$ 5,13 trilhões em 2022 para projeções de US\$ 6,56 trilhões em 2025, com expectativa de US\$ 8,09 trilhões até 2028 \cite{shopify_relatorio_2025}. Esse crescimento impõe desafios técnicos às organizações que desenvolvem e mantêm plataformas de \textit{e-commerce}, particularmente em desempenho, escalabilidade, experiência do usuário e conversão de vendas.

Decisões técnicas no desenvolvimento de plataformas de \textit{e-commerce} impactam diretamente sucesso e receita das empresas. Desde escolhas arquiteturais padrões de design, distribuição de carga, escalabilidade  até implementações de interface, inteligência artificial e otimizações de performance, cada decisão possui implicações mensuráveis no comportamento do usuário e resultados financeiros.

A diversificação dos modelos de monetização, incluindo pagamento único e assinaturas recorrentes, impõe desafios relacionados à maximização de receita recorrente (\textit{MRR}), aumento do valor do tempo de vida do cliente (\textit{CLTV}) e redução de \textit{churn} (taxa de cancelamento). Estratégias técnicas de \textit{upsell} (venda de produtos/planos superiores), ofertas de retenção no cancelamento e personalização por perfil de assinante representam oportunidades relevantes, mas permanecem pouco exploradas na literatura.

Este trabalho investiga o estado da arte no desenvolvimento de plataformas de \textit{e-commerce}, focando em três dimensões: (i) estratégias arquiteturais e de performance para escalabilidade; (ii) abordagens de \textit{UX/UI} para maximização de conversões; e (iii) integração de inteligência artificial para personalização e automação.

A relevância deste estudo está na necessidade de compreender a relação entre decisões técnicas e impacto financeiro. A literatura apresenta estudos isolados sobre performance arquitetural, otimização \textit{mobile} e sistemas de recomendação, mas há espaço para análise integrada dessas dimensões e sua relação com métricas de receita.




