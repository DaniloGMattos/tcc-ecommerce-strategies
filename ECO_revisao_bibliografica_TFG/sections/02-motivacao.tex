\newpage
\section{Motivação}
\label{sc:motivacao}


Este trabalho é motivado pela lacuna entre conhecimento técnico disponível e sua aplicação estratégica com foco no impacto financeiro de plataformas de \textit{e-commerce}. Três aspectos justificam este estudo.

\textbf{Impacto econômico de decisões técnicas.} O desenvolvimento de plataformas de \textit{e-commerce} envolve investimentos em infraestrutura, arquitetura de software e recursos humanos. Decisões técnicas inadequadas resultam em perdas de receita mensuráveis. Casos documentados demonstram impacto direto: a implementação de \textit{Progressive Web App} no \textit{Alibaba.com} resultou em aumento de 76\% nas conversões \cite{noauthor_alibaba_nodate}, evidenciando as implicações financeiras de escolhas arquiteturais e de \textit{performance}.

\textbf{Complexidade do ecossistema tecnológico.} O desenvolvimento de \textit{e-commerce} apresenta diversidade de abordagens: padrões arquiteturais, \textit{Progressive Web Apps} versus aplicações tradicionais, sistemas de recomendação por filtragem colaborativa versus \textit{deep learning}. Cada decisão envolve \textit{trade-offs} (equilibrar ganhos e perdas) em custo, complexidade, manutenibilidade e performance. Empresas necessitam orientação baseada em evidências para tomar decisões que maximizem retorno sobre investimento.

\textbf{Lacuna entre pesquisa e prática.} A literatura oferece estudos sobre performance arquitetural \cite{ubur_reviewing_2023, zhao_systematic_2024}, otimizações de \textit{UX/UI} \cite{noauthor_pdf_nodate, noauthor_pdf_2025-1} e algoritmos de recomendação \cite{nguyen_personalized_2024}, mas carece de trabalhos que integrem essas dimensões e analisem impacto conjunto na receita. Casos como o \textit{Alibaba.com}, com aumento de 76\% nas conversões após implementação de \textit{PWA} \cite{noauthor_alibaba_nodate}, demonstram potencial de inovações técnicas, mas faltam estudos que sistematizem experiências e identifiquem padrões replicáveis.

Este trabalho busca endereçar essas lacunas, conectando fundamentos técnicos a resultados de negócio. Adicionalmente, insere-se no contexto de formação em Engenharia de Computação da UNIFEI, integrando conhecimentos técnicos com compreensão de seu potencial impacto organizacional e econômico.
