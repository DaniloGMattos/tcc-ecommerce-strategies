\newpage
\section{Objetivos específicos}
\label{sc:objetivos}


O objetivo geral deste trabalho é analisar o cenário atual do desenvolvimento de plataformas de \textit{e-commerce}, identificando e caracterizando as principais estratégias técnicas empregadas e avaliando seu impacto na receita e no sucesso das empresas. Para alcançar este objetivo geral, foram definidos os seguintes objetivos específicos:

\begin{enumerate}
    \item \textbf{Mapear as estratégias arquiteturais}: Identificar e caracterizar as principais abordagens arquiteturais empregadas no desenvolvimento de plataformas de \textit{e-commerce} modernas, analisando os \textit{trade-offs} entre diferentes padrões de design, estratégias de distribuição e modelos de escalabilidade em termos de desempenho, custo de desenvolvimento e capacidade de adaptação a diferentes contextos de negócio.

    \item \textbf{Analisar estratégias de otimização de experiência do usuário}: Examinar as principais técnicas e tecnologias utilizadas para otimização da experiência do usuário em plataformas de \textit{e-commerce}, com ênfase em otimizações \textit{mobile}, \textit{Progressive Web Apps} (\textit{PWAs}), design responsivo e otimizações de performance, relacionando essas estratégias com métricas de conversão e satisfação do cliente.

    \item \textbf{Avaliar o papel da inteligência artificial}: Investigar a aplicação de técnicas de inteligência artificial e aprendizado de máquina em plataformas de \textit{e-commerce}, incluindo sistemas de recomendação, personalização de conteúdo e geração de conteúdo via IA generativa, analisando a eficácia dessas implementações e os desafios relacionados à aceitação dos usuários, privacidade e segurança.

    \item \textbf{Correlacionar decisões técnicas com impacto na receita}: Identificar e analisar casos documentados que demonstrem relação mensurável entre decisões técnicas específicas (arquitetura, \textit{UX/UI}, IA, estratégias de marketing) e métricas de negócio (taxas de conversão, receita, satisfação do cliente, retenção), estabelecendo quando possível relações quantitativas entre estratégias técnicas e resultados financeiros, considerando tanto modelos de pagamento único quanto assinaturas recorrentes.

    \item \textbf{Identificar lacunas e oportunidades}: Mapear lacunas no conhecimento atual sobre desenvolvimento de plataformas de \textit{e-commerce} e identificar oportunidades para pesquisa futura, contribuindo para o avanço do campo e para orientação de decisões estratégicas em organizações que desenvolvem ou operam plataformas de comércio eletrônico.
\end{enumerate}


