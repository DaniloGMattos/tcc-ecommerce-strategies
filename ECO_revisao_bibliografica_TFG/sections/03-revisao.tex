\newpage
\section{Revisão bibliográfica}
\label{sc:revisao}





Esta seção apresenta uma revisão crítica da literatura relacionada ao cenário atual do desenvolvimento de plataformas de \textit{e-commerce}, com foco particular nas estratégias técnicas empregadas e seu impacto mensurável na receita e sucesso das empresas. A revisão está organizada em quatro eixos temáticos: (i) arquiteturas de software e desempenho, (ii) experiência do usuário e otimizações de interface, (iii) inteligência artificial e personalização, e (iv) métricas de conversão e análise de tráfego. Cada subseção analisa as estratégias técnicas adotadas, os desafios enfrentados e, quando disponível, o impacto quantitativo dessas decisões na receita e conversão das plataformas.

\subsection{Arquiteturas de Software e Desempenho em \textit{E-commerce}}

A escolha arquitetural de sistemas de \textit{e-commerce} impacta diretamente desempenho e escalabilidade. Ubur \cite{ubur_reviewing_2023} comparou sistemas monolíticos e arquiteturas distribuídas orientadas a eventos, desenvolvendo protótipos e avaliando-os com \textit{Dropwizard Metrics} e \textit{Apache JMeter}. Os resultados mostraram que padrões arquiteturais apresentam desempenho variado conforme contexto: sistemas centralizados oferecem melhores tempos de resposta em cenários de baixa complexidade, enquanto arquiteturas distribuídas performam melhor com alta complexidade e volume de requisições. A escolha depende do contexto técnico e necessidades da plataforma.

Zhao et al. \cite{zhao_systematic_2024} mapearam 109 estudos sobre abordagens arquiteturais para análise de desempenho, identificando que a integração entre arquitetura e performance carece de ferramentas e \textit{benchmarks} para replicação e comparação. Os autores destacam três lacunas: (i) técnicas para domínios emergentes, (ii) consideração de fatores práticos de adoção, e (iii) integração de aprendizado de máquina para análise eficiente.

O uso de servidores distribuídos para \textit{e-commerce} \cite{noauthor_performance_nodate} demonstra como distribuição de carga pode melhorar desempenho em cenários de alta demanda. Essa abordagem oferece resiliência e escalabilidade, características essenciais para crescimento sustentável de receita.

\subsection{Experiência do Usuário e Otimizações de Interface}

\textit{UX} e \textit{UI} desempenham papel fundamental nas taxas de conversão de plataformas de \textit{e-commerce}. A pesquisa sobre impacto de design \textit{UX/UI} nas conversões \cite{noauthor_pdf_nodate} enfatiza que experiência fluida e intuitiva é determinante para sucesso. O estudo identifica que navegação intuitiva, carregamento rápido e design responsivo impactam diretamente satisfação e conclusão de transações.

Complementando essa perspectiva, o estudo sobre o impacto da usabilidade de \textit{websites} e otimização \textit{mobile} na satisfação do cliente e taxas de conversão em negócios de \textit{e-commerce} na Indonésia \cite{noauthor_pdf_2025-1} investigou empiricamente como a usabilidade do site e a otimização para dispositivos móveis afetam a satisfação do cliente e as taxas de conversão de vendas. Os resultados indicam uma correlação positiva significativa entre usabilidade, otimização \textit{mobile} e satisfação do cliente, reforçando a importância de priorizar a experiência em dispositivos móveis, especialmente considerando a crescente predominância do acesso \textit{mobile} no comércio eletrônico.

A otimização de \textit{SEO} (\textit{Search Engine Optimization}) também contribui para a experiência do usuário ao melhorar os tempos de carregamento das páginas. O trabalho sobre como \textit{SEO} torna o carregamento de \textit{websites} mais rápido e auxilia no engajamento de usuários \cite{noauthor_pdf_2025} demonstra que práticas adequadas de \textit{SEO} não apenas melhoram a visibilidade nos mecanismos de busca, mas também resultam em páginas mais leves e rápidas, contribuindo para melhor retenção e engajamento de usuários.

Um exemplo concreto de implementação bem-sucedida é apresentado pelo caso de estudo do \textit{Alibaba.com} \cite{noauthor_alibaba_nodate}, a maior plataforma \textit{B2B} \textit{online} do mundo. Após a implementação de um \textit{Progressive Web App} (\textit{PWA}), a empresa registrou um aumento de 76\% no total de conversões em todos os navegadores. Este caso demonstra o potencial de tecnologias \textit{web} modernas para melhorar simultaneamente a experiência do usuário e as métricas de negócio. A coleção de estudos de caso sobre \textit{PWAs} \cite{noauthor_pwa_nodate} apresenta diversos outros exemplos que corroboram esses benefícios.

\subsection{Inteligência Artificial e Personalização}

A integração de inteligência artificial em plataformas de \textit{e-commerce} tem transformado a interação entre empresas e clientes. Stamkou et al. \cite{stamkou_user_2025} avaliaram a percepção de usuários sobre conteúdo gerado por IA através de questionário aplicado a 223 participantes. Os autores desenvolveram uma loja \textit{online} utilizando \textit{ChatGPT} e \textit{DALL·E} para geração automática de conteúdo, avaliando aspectos de funcionalidade, estética e segurança. Por meio de análise fatorial exploratória (\textit{EFA}), identificaram dois componentes que influenciam a experiência: ``Qualidade de Serviço e Segurança'' e ``\textit{Design} e Estética''. Os participantes avaliaram positivamente o conteúdo gerado por IA, embora demonstrassem cautela quanto à segurança. Os resultados sugerem que IA generativa pode ser ferramenta relevante para desenvolvimento de plataformas de \textit{e-commerce}.

Uma revisão sistemática mais abrangente sobre inteligência artificial em \textit{e-commerce} e \textit{marketing} digital \cite{noauthor_pdf_2025-2} examinou oportunidades, desafios e implicações éticas da aplicação de IA nestes domínios. O estudo enfatiza que, embora a IA ofereça oportunidades significativas para personalização e automação, também apresenta desafios relacionados à privacidade de dados, viés algorítmico e transparência nas decisões automatizadas.

No contexto de sistemas de recomendação, Nguyen et al. \cite{nguyen_personalized_2024} propuseram um modelo de recomendação personalizada baseado em estratégia de recuperação, desenvolvido em colaboração com o Grupo H\&M. O modelo combina filtragem colaborativa, popularidade e \textit{ranking} personalizado bayesiano. Os autores compararam duas técnicas de aprendizado de máquina para avaliação de candidatos: \textit{LightGBM} e Redes Neurais Profundas. O modelo \textit{LightGBM} apresentou desempenho superior, com \textit{MAP@50} de 0,06 contra 0,02 das redes neurais, e \textit{MAR@50} de 0,03 contra 0,01. O \textit{framework} proposto aborda desafios de análise de dados em larga escala e problemas de \textit{cold-start}, contribuindo para melhorar a experiência do usuário em plataformas de \textit{e-commerce}.

\subsection{Métricas de Conversão e Análise de Tráfego}

Métricas de conversão e análise de fontes de tráfego são fundamentais para otimização de plataformas de \textit{e-commerce}. Muralidhar e Lakkanna \cite{muralidhar_clicks_2024} analisaram interações de usuários, métricas de conversão e jornada completa do usuário em uma plataforma de \textit{e-commerce}. O trabalho examinou taxas de saída e sessões em diferentes dispositivos e navegadores, taxas de conversão por fonte de tráfego, e a jornada desde visualização do produto até o \textit{checkout}.

Os resultados indicam necessidade de melhorias em otimização \textit{mobile} e compatibilidade entre navegadores, dado que dispositivos móveis apresentaram taxas de saída mais altas. A análise por fonte de tráfego revelou efetividade variável: mídias anunciadas demonstraram maior potencial, enquanto tráfego de referência e afiliados apresentaram desempenho inferior. O exame da jornada do usuário identificou gargalos no processo de conversão, com lacuna entre interesse e transações completadas. Os autores recomendam melhorias no processo de \textit{checkout} para otimizar conversões.

Em modelos de assinatura, estratégias de \textit{upsell} e retenção são relevantes para maximização de receita recorrente e redução de \textit{churn}. A literatura apresenta análises sobre conversão em modelos transacionais, mas carece de estudos sistemáticos sobre implementação técnica de estratégias como ofertas de \textit{upsell} na jornada de compra, ofertas de retenção no fluxo de cancelamento (\textit{cancellation offers}) e \textit{timing} de apresentação. Em plataformas híbridas, a análise de métricas como \textit{MRR}, \textit{CLTV}, taxa de \textit{churn} e efetividade de estratégias de retenção permanece como área relevante para investigação.

\subsection{Síntese: Estratégias Técnicas e Impacto na Receita}

A literatura revisada demonstra que o cenário atual do desenvolvimento de plataformas de \textit{e-commerce} é caracterizado pela convergência de múltiplas estratégias técnicas que impactam diretamente a receita e o sucesso empresarial. Esta síntese organiza os achados principais relacionando estratégias técnicas com seu impacto mensurável.

\textbf{Decisões arquiteturais e escalabilidade.} As decisões arquiteturais têm impacto direto na capacidade de plataformas de \textit{e-commerce} suportarem crescimento e escalabilidade. O estudo de Ubur (2023) demonstrou que diferentes abordagens arquiteturais apresentam vantagens contextuais: sistemas mais centralizados oferecem benefícios em cenários de baixa complexidade, enquanto arquiteturas distribuídas demonstram melhor desempenho em contextos de alta demanda. Esta constatação evidencia que não existe uma solução arquitetural universalmente superior; a escolha adequada depende do contexto específico do negócio, volume esperado de transações, padrões de crescimento e recursos disponíveis. A escolha arquitetural, portanto, possui implicações diretas na capacidade de uma plataforma suportar crescimento de receita sem degradação de performance.

\textbf{Experiência do usuário como fator crítico de conversão.} A experiência do usuário, especialmente em dispositivos móveis, é consistentemente identificada como fator crítico para taxas de conversão e, consequentemente, para a receita. O caso do \textit{Alibaba.com} exemplifica este impacto de forma quantitativa: a implementação de \textit{PWA} resultou em aumento de 76\% nas conversões totais,  tradução direta de decisão técnica em crescimento de receita. A literatura revisada indica correlação positiva significativa entre usabilidade, otimização \textit{mobile} e satisfação do cliente, com impacto mensurável nas taxas de conversão de vendas.

\textbf{Personalização via IA: oportunidades e desafios.} A integração de inteligência artificial oferece oportunidades substanciais para personalização e automação, com sistemas de recomendação demonstrando eficácia mensurável. O \textit{framework} desenvolvido por Nguyen et al. (2024) para o Grupo H\&M ilustra como técnicas de aprendizado de máquina podem contribuir para melhorar a experiência do usuário e impulsionar vendas. No entanto, a literatura também evidencia desafios relacionados à aceitação dos usuários, especialmente em questões de segurança e privacidade, que podem impactar negativamente a confiança e, consequentemente, as taxas de conversão.

\textbf{Otimização de métricas como estratégia contínua.} A análise de Muralidhar e Lakkanna (2024) sobre a jornada do usuário e métricas de conversão destaca que o desenvolvimento de plataformas de \textit{e-commerce} não é um processo estático, mas requer otimização contínua baseada em dados. A identificação de gargalos no processo de \textit{checkout}, variações nas taxas de conversão por fonte de tráfego e diferenças de performance entre dispositivos e navegadores representam oportunidades concretas para incremento de receita através de melhorias técnicas direcionadas.

\subsubsection{Oportunidades de Pesquisa}

Apesar dos avanços documentados, ainda há oportunidades de pesquisa sobre estratégias técnicas e seu impacto na receita de plataformas de \textit{e-commerce}:

\begin{enumerate}
    \item \textbf{\textit{Frameworks} integrados de análise:} A literatura apresenta estudos isolados sobre aspectos específicos de arquitetura \cite{ubur_reviewing_2023, zhao_systematic_2024}, \textit{UX} \cite{noauthor_pdf_nodate, noauthor_pdf_2025-1} e IA \cite{stamkou_user_2025, nguyen_personalized_2024}, mas carece de \textit{frameworks} integrados que combinem análise arquitetural, desempenho, experiência do usuário e métricas de negócio de forma holística. Esta lacuna dificulta a tomada de decisões estratégicas que considere \textit{trade-offs} entre diferentes dimensões técnicas.

    \item \textbf{Estudos de longo prazo sobre \textit{ROI} de implementações de IA:} Embora estudos demonstrem a eficácia técnica de sistemas de IA e recomendação \cite{stamkou_user_2025, nguyen_personalized_2024, noauthor_pdf_2025-2}, há carência de estudos de longo prazo que avaliem o retorno sobre investimento dessas implementações, considerando custos de desenvolvimento, manutenção e impacto em métricas de satisfação, fidelização e \textit{lifetime value} de clientes.

    \item \textbf{\textit{Benchmarks} padronizados para comparação arquitetural:} A ausência de \textit{benchmarks} padronizados e \textit{datasets} de referência para avaliação de diferentes abordagens arquiteturais em contextos de \textit{e-commerce}, conforme identificado por Zhao et al. \cite{zhao_systematic_2024}, dificulta comparações objetivas e decisões baseadas em evidências sobre escolhas arquiteturais.

    \item \textbf{Otimização integrada via \textit{ML/AI}:} A exploração de técnicas modernas de aprendizado de máquina para otimização integrada e automática de múltiplos aspectos de plataformas de \textit{e-commerce} (arquitetura, performance, \textit{UX}, personalização) permanece limitada, representando oportunidade significativa para pesquisa futura.

    \item \textbf{Relação quantitativa entre decisões técnicas e receita:} Embora casos como o \textit{Alibaba.com} \cite{noauthor_alibaba_nodate} forneçam dados quantitativos sobre impacto de implementações específicas, faltam estudos sistemáticos que estabeleçam relações causais e quantitativas entre decisões técnicas específicas (por exemplo, tempo de resposta de \textit{APIs}, qualidade de recomendações, design de \textit{checkout}) e métricas de receita, considerando diferentes contextos de negócio e segmentos de mercado.

    \item \textbf{Estratégias de monetização via assinatura:} Há carência de estudos que investiguem sistematicamente a implementação técnica de estratégias de \textit{upsell}, retenção e ofertas de cancelamento em modelos de \textit{e-commerce} baseados em assinatura. Pesquisas futuras devem explorar o impacto de diferentes abordagens de \textit{timing}, \textit{messaging} e tipo de oferta em métricas críticas como \textit{MRR}, \textit{CLTV}, taxa de \textit{churn} e receita incremental, especialmente em plataformas híbridas que combinam pagamento único e assinaturas recorrentes.
\end{enumerate}
