\documentclass{beamer}
\usepackage[en]{zafu}
\usepackage[utf8]{inputenc}
\usepackage[T1]{fontenc}

\definecolor{hrefcol}{RGB}{0, 58, 112}

% meta-data
\title{Cenário Atual do Desenvolvimento de Plataformas de E-commerce}
\subtitle{Estratégias Técnicas e Impacto na Receita das Empresas}
\author{Danilo Godofredo de Mattos\\Orientadora: Profa. Bárbara Pimenta Caetano}
\date{2025}

% document body
\begin{document}

    \maketitle

    % SLIDE 1: Contexto e Problema
    \begin{frame}{Contexto e Problema}
        \textbf{Crescimento do E-commerce:}
        \begin{itemize}
            \item Vendas globais: US\$ 6,56 trilhões (2025) $\rightarrow$ US\$ 8,09 trilhões (2028)
            \item Desafios técnicos: performance, escalabilidade, experiência do usuário
        \end{itemize}

        \vspace{0.5cm}

        \textbf{O Problema:}
        \begin{itemize}
            \item Informação técnica dispersa na literatura
            \item Falta visão integrada: arquitetura + UX + IA
            \item Falta conexão entre decisões técnicas e impacto financeiro
        \end{itemize}

        \vspace{0.5cm}

        \textbf{Exemplo Concreto:}
        \begin{itemize}
            \item Alibaba.com: implementação de PWA $\rightarrow$ \textbf{+76\% conversões}
            \item Decisões técnicas têm impacto direto na receita
        \end{itemize}
    \end{frame}

    % SLIDE 2: Motivação e Objetivos
    \begin{frame}{Motivação e Objetivos}
        \textbf{Motivação:}
        \begin{itemize}
            \item Impacto econômico direto de decisões técnicas
            \item Complexidade das escolhas: arquitetura, PWA, algoritmos de IA
            \item Lacuna entre pesquisa acadêmica e prática empresarial
        \end{itemize}

        \vspace{0.3cm}

        \textbf{Objetivo Geral:}
        \begin{itemize}
            \item Analisar estratégias técnicas em e-commerce e avaliar impacto na receita
        \end{itemize}

        \vspace{0.3cm}

        \textbf{Objetivos Específicos:}
        \begin{enumerate}
            \item Mapear estratégias arquiteturais e trade-offs
            \item Analisar estratégias de UX/UI para conversão
            \item Avaliar papel da IA e ML
            \item \textbf{Correlacionar decisões técnicas com receita}
            \item Identificar lacunas e oportunidades de pesquisa
        \end{enumerate}
    \end{frame}

    % SLIDE 3: Exemplo Prático
    \begin{frame}{Exemplo Prático}
        \textbf{Plataforma Achievece:}
        \begin{itemize}
            \item E-commerce de cursos online (educação médica, EUA)
            \item 30 mil visitantes/mês, receita: US\$ 2.500/dia
            \item Modelo híbrido: pagamento único + assinaturas
        \end{itemize}

        \vspace{0.3cm}

        \textbf{Estrutura dos Testes (A/B Testing):}
        \begin{itemize}
            \item Divisão aleatória de tráfego
            \item Significância estatística (95\% confiança)
            \item Ferramentas: GrowthBook, Google Analytics, Lighthouse
        \end{itemize}

        \vspace{0.3cm}

        \textbf{Dimensões de Testes:}
        \begin{enumerate}
            \item \textbf{Performance:} Core Web Vitals (LCP, FID, CLS)
            \item \textbf{UX:} Redesign checkout, formulários, indicadores
            \item \textbf{Retenção:} Upsell, ofertas de cancelamento, timing
        \end{enumerate}

        \vspace{0.2cm}

        \textbf{Métricas:} Taxa de conversão, MRR, CLTV, churn, RPV
    \end{frame}

    % SLIDE 4: Cronograma
    \begin{frame}{Cronograma e Considerações Finais}
        \textbf{Cronograma 2025:}
        \begin{itemize}
            \item \textbf{1º Trimestre:} Revisão bibliográfica, protocolos de testes
            \item \textbf{2º Trimestre:} Execução dos testes, coleta de dados
            \item \textbf{3º Trimestre:} Análise estatística, escrita do trabalho
            \item \textbf{4º Trimestre:} Revisão, ajustes finais, defesa
        \end{itemize}

        \vspace{0.5cm}

        \textbf{Expectativa:}
        \begin{itemize}
            \item Apresentar possíveis soluções técnicas que possam contribuir para o aumento da receita de plataformas de e-commerce
            \item Oferecer orientação prática baseada em evidências da literatura e exemplo real
        \end{itemize}

        \vspace{0.5cm}

        \centering
        \textbf{Obrigado pela atenção!}
    \end{frame}

    % SLIDE 5: Revisão Bibliográfica
    \begin{frame}{Revisão Bibliográfica}
        \textbf{12 artigos recentes (2023-2025) em 4 eixos temáticos:}

        \vspace{0.3cm}

        \textbf{1. Arquitetura e Performance}
        \begin{itemize}
            \item Ubur (2023): event-driven vs monolítico
            \item Zhao et al. (2024): mapeamento de 109 estudos
            \item Contexto determina a melhor arquitetura
        \end{itemize}

        \textbf{2. UX e Interface}
        \begin{itemize}
            \item Impacto de UX/UI nas conversões
            \item Otimização mobile e SEO
            \item Casos PWA: Alibaba.com (+76\%)
        \end{itemize}

        \textbf{3. Inteligência Artificial}
        \begin{itemize}
            \item IA generativa (Stamkou et al., 2025)
            \item Sistema H\&M: LightGBM (MAP@50=0,06) vs DNN (0,02)
        \end{itemize}

        \textbf{4. Métricas e Conversão}
        \begin{itemize}
            \item Análise de jornada do usuário (Muralidhar \& Lakkanna, 2024)
            \item Gargalos no checkout e otimização por fonte de tráfego
        \end{itemize}
    \end{frame}

    % SLIDE 6: Resultados Esperados
    \begin{frame}{Resultados Esperados}
        \textbf{1. Framework de Análise Integrada}
        \begin{itemize}
            \item Conectar arquitetura + UX + IA com métricas de negócio
            \item Orientar decisões baseadas em evidências
        \end{itemize}

        \vspace{0.3cm}

        \textbf{2. Evidências Quantitativas}
        \begin{itemize}
            \item Impacto real de otimizações (ex: LCP 3s $\rightarrow$ 1.5s)
            \item Efetividade de estratégias de retenção em assinaturas
            \item Relação entre Core Web Vitals e conversão
        \end{itemize}

        \vspace{0.3cm}

        \textbf{3. Identificação de Lacunas}
        \begin{itemize}
            \item Falta de frameworks integrados
            \item Estudos de longo prazo sobre ROI de IA
            \item Benchmarks arquiteturais padronizados
            \item Estratégias de monetização via assinatura
            \item Relação quantitativa: decisões técnicas $\leftrightarrow$ receita
        \end{itemize}
    \end{frame}

    \QApage

\end{document}
