%% abtex2-modelo-trabalho-academico.tex, v-1.9.6 laurocesar
%% Copyright 2012-2016 by abnTeX2 group at http://www.abntex.net.br/
%%
%% This work may be distributed and/or modified under the
%% conditions of the LaTeX Project Public License, either version 1.3
%% of this license or (at your option) any later version.
%% The latest version of this license is in
%%   http://www.latex-project.org/lppl.txt
%% and version 1.3 or later is part of all distributions of LaTeX
%% version 2005/12/01 or later.
%%
%% This work has the LPPL maintenance status `maintained'.
%%
%% The Current Maintainer of this work is the abnTeX2 team, led
%% by Lauro César Araujo. Further information are available on
%% http://www.abntex.net.br/
%%
%% This work consists of the files abntex2-modelo-trabalho-academico.tex,
%% abntex2-modelo-include-comandos and abntex2-modelo-references.bib
%%

% ------------------------------------------------------------------------
% ------------------------------------------------------------------------
% abnTeX2: Modelo de Trabalho Academico (tese de doutorado, dissertacao de
% mestrado e trabalhos monograficos em geral) em conformidade com
% ABNT NBR 14724:2011: Informacao e documentacao - Trabalhos academicos -
% Apresentacao
% ------------------------------------------------------------------------
% ------------------------------------------------------------------------

\documentclass[
	% -- opções da classe memoir --
	12pt,				% tamanho da fonte
%	openright,			% capítulos começam em pág ímpar (insere página vazia caso preciso)
	twoside,			% para impressão em recto e verso. Oposto a oneside
%    oneside,			% Oposto a twoside
	a4paper,			% tamanho do papel.
	% -- opções da classe abntex2 --
	%chapter=TITLE,		% títulos de capítulos convertidos em letras maiúsculas
	%section=TITLE,		% títulos de seções convertidos em letras maiúsculas
	%subsection=TITLE,	% títulos de subseções convertidos em letras maiúsculas
	%subsubsection=TITLE,% títulos de subsubseções convertidos em letras maiúsculas
	% -- opções do pacote babel --
	english,			% idioma adicional para hifenização
	french,				% idioma adicional para hifenização
	spanish,			% idioma adicional para hifenização
	brazil				% o último idioma é o principal do documento
	]{abntex2}

% ---
% Pacotes 
% ---
\usepackage{lmodern}			% Usa a fonte Latin Modern			
\usepackage[T1]{fontenc}		% Selecao de codigos de fonte.
\usepackage[utf8]{inputenc}		% Codificacao do documento (conversão automática dos acentos)
\usepackage{lastpage}			% Usado pela Ficha catalográfica
\usepackage{indentfirst}		% Indenta o primeiro parágrafo de cada seção.
\usepackage{color}				% Controle das cores
\usepackage{graphicx}			% Inclusão de gráficos
\usepackage{microtype} 			% para melhorias de justificação

\usepackage{lipsum}				% para geração de dummy text

\usepackage[brazilian,hyperpageref]{backref}	 % Paginas com as citações na bibl
\usepackage[alf]{abntex2cite}	% Citações padrão ABNT

% ---
% CONFIGURAÇÕES DE PACOTES
% ---

% ---
% Configurações do pacote backref
% Usado sem a opção hyperpageref de backref
\renewcommand{\backrefpagesname}{Citado na(s) página(s):~}
% Texto padrão antes do número das páginas
\renewcommand{\backref}{}
% Define os textos da citação
\renewcommand*{\backrefalt}[4]{
	\ifcase #1 %
		Nenhuma citação no texto.%
	\or
		Citado na página #2.%
	\else
		Citado #1 vezes nas páginas #2.%
	\fi}%
% ---

% ---
% Informações de dados para CAPA e FOLHA DE ROSTO
% ---
\titulo{Cenário Atual do Desenvolvimento de Plataformas de E-commerce: \\ Estratégias Técnicas e Impacto na Receita das Empresas}
\autor{Danilo Godofredo de Mattos}
\local{Itajubá -- MG}
\data{2025}
\orientador{Profa. Bárbara Pimenta Caetano}
%\coorientador{}
\instituicao{%
  Universidade Federal de Itajubá -- UNIFEI
  \par
  Instituto de Engenharia de Sistemas e Tecnologia da Informação
  \par
  Engenharia da Computação}
\tipotrabalho{Trabalho Final de Graduação}
% O preambulo deve conter o tipo do trabalho, o objetivo,
% o nome da instituição e a área de concentração
\preambulo{Trabalho Final de Graduação apresentado ao curso de Engenharia de Computação da Universidade Federal de Itajubá como requisito parcial para obtenção do título de Bacharel em Engenharia de Computação.}
% ---


% ---
% Configurações de aparência do PDF final

% alterando o aspecto da cor azul
\definecolor{blue}{RGB}{41,5,195}

% informações do PDF
\makeatletter
\hypersetup{
     	%pagebackref=true,
		pdftitle={\@title},
		pdfauthor={\@author},
    	pdfsubject={\imprimirpreambulo},
	    pdfcreator={LaTeX with abnTeX2},
		pdfkeywords={e-commerce}{arquitetura de software}{inteligência artificial}{experiência do usuário}{conversão},
		colorlinks=true,       		% false: boxed links; true: colored links
    	linkcolor=blue,          	% color of internal links
    	citecolor=blue,        		% color of links to bibliography
    	filecolor=magenta,      		% color of file links
		urlcolor=blue,
		bookmarksdepth=4
}
\makeatother
% ---

% ---
% Espaçamentos entre linhas e parágrafos
% ---

% O tamanho do parágrafo é dado por:
\setlength{\parindent}{1.3cm}

% Controle do espaçamento entre um parágrafo e outro:
\setlength{\parskip}{0.2cm}  % tente também \onelineskip

% ---
% compila o indice
% ---
\makeindex
% ---

% ----
% Início do documento
% ----
\begin{document}

% Seleciona o idioma do documento (conforme pacotes do babel)
%\selectlanguage{english}
\selectlanguage{brazil}

% Retira espaço extra obsoleto entre as frases.
\frenchspacing

% ----------------------------------------------------------
% ELEMENTOS PRÉ-TEXTUAIS
% ----------------------------------------------------------
% Capa e Folha de Rosto
\imprimircapa
\imprimirfolhaderosto*
% ---

% ---
% Dedicatória
% ---
\begin{dedicatoria}
   \vspace*{\fill}
   \centering
   \noindent
   \textit{À minha família, pelo apoio incondicional.} \vspace*{\fill}
\end{dedicatoria}
% ---

% ---
% Agradecimentos
% ---
\begin{agradecimentos}
Agradeço primeiramente à minha família pelo suporte e incentivo durante toda a jornada acadêmica.

À Profa. Bárbara Pimenta Caetano, pela orientação, paciência e contribuições fundamentais para a realização deste trabalho.

Aos professores do curso de Engenharia de Computação da UNIFEI, pelos ensinamentos e pela formação técnica que recebi ao longo desses anos.

Aos colegas de curso, pelas discussões, trocas de experiência e companheirismo durante a graduação.

A todos que, direta ou indiretamente, contribuíram para a conclusão deste trabalho.
\end{agradecimentos}
% ---

% ---
% Epígrafe (opcional - comentado por enquanto)
% ---
% \begin{epigrafe}
%     \vspace*{\fill}
% 	\begin{flushright}
% 		\textit{``Texto da epígrafe aqui.\\
% 		(Autor)}
% 	\end{flushright}
% \end{epigrafe}
% ---

% ---
% RESUMOS
% ---

% resumo em português
\setlength{\absparsep}{18pt} % ajusta o espaçamento dos parágrafos do resumo
\begin{resumo}
O comércio eletrônico apresenta crescimento acelerado, com projeções de US\$ 8,09 trilhões em vendas globais até 2028. As decisões técnicas no desenvolvimento de plataformas de e-commerce impactam diretamente o sucesso e a receita das empresas. Este trabalho analisa o cenário atual do desenvolvimento de plataformas de e-commerce, identificando e caracterizando as principais estratégias técnicas empregadas. A pesquisa baseia-se em revisão bibliográfica sistemática, contemplando três dimensões principais: estratégias arquiteturais e de performance para escalabilidade; abordagens de experiência do usuário (UX/UI) para maximização de conversões; e integração de inteligência artificial para personalização e automação. Os resultados evidenciam que decisões arquiteturais (monolítica versus microsserviços), otimizações de interface (Progressive Web Apps), e sistemas de recomendação baseados em IA possuem correlação mensurável com métricas de receita. O trabalho também identifica lacunas na literatura quanto à relação quantitativa entre decisões técnicas específicas e impacto financeiro, sugerindo oportunidades para estudos futuros sobre frameworks integrados de análise e benchmarks padronizados.

 \textbf{Palavras-chave}: e-commerce. arquitetura de software. inteligência artificial. experiência do usuário. conversão.
\end{resumo}

% resumo em inglês
\begin{resumo}[Abstract]
 \begin{otherlanguage*}{english}
   E-commerce shows accelerated growth, with projections of US\$ 8.09 trillion in global sales by 2028. Technical decisions in e-commerce platform development directly impact business success and revenue. This work analyzes the current scenario of e-commerce platform development, identifying and characterizing the main technical strategies employed. The research is based on systematic literature review, covering three main dimensions: architectural and performance strategies for scalability; user experience (UX/UI) approaches to maximize conversions; and artificial intelligence integration for personalization and automation. Results show that architectural decisions (monolithic versus microservices), interface optimizations (Progressive Web Apps), and AI-based recommendation systems have measurable correlation with revenue metrics. The work also identifies gaps in literature regarding the quantitative relationship between specific technical decisions and financial impact, suggesting opportunities for future studies on integrated analysis frameworks and standardized benchmarks.

   \vspace{\onelineskip}

   \noindent
   \textbf{Keywords}: e-commerce. software architecture. artificial intelligence. user experience. conversion.
 \end{otherlanguage*}
\end{resumo}

% ---
% inserir lista de ilustrações
% ---
\pdfbookmark[0]{\listfigurename}{lof}
\listoffigures*
\clearpage
% ---

% ---
% inserir lista de tabelas
% ---
\pdfbookmark[0]{\listtablename}{lot}
\listoftables*
\clearpage
% ---

% ---
% inserir lista de abreviaturas e siglas
% ---
\begin{siglas}
  \item[ABNT] Associação Brasileira de Normas Técnicas
  \item[API] Application Programming Interface
  \item[CLTV] Customer Lifetime Value
  \item[IA] Inteligência Artificial
  \item[ML] Machine Learning
  \item[MRR] Monthly Recurring Revenue
  \item[PWA] Progressive Web App
  \item[SEO] Search Engine Optimization
  \item[UX] User Experience
  \item[UI] User Interface
\end{siglas}
% ---

% ---
% inserir o sumario
% ---
\pdfbookmark[0]{\contentsname}{toc}
\tableofcontents*
\clearpage
% ---

% ----------------------------------------------------------
% ELEMENTOS TEXTUAIS
% ----------------------------------------------------------
\textual

% ----------------------------------------------------------------
% Capítulos
% ----------------------------------------------------------------
% ----------------------------------------------------------------
% Introdução
% ----------------------------------------------------------------
\chapter{Introdução}
\label{ch:introducao}

\section{Contexto e Relevância}

O comércio eletrônico tem se consolidado como uma das principais formas de transação comercial no mundo. Segundo dados de \cite{shopify_relatorio_2025}, as vendas globais de e-commerce evoluíram de US\$ 5,13 trilhões em 2022 para projeções de US\$ 6,56 trilhões em 2025, com expectativa de atingir US\$ 8,09 trilhões até 2028. Esse crescimento traz consigo uma série de desafios técnicos para as empresas que desenvolvem e mantêm plataformas de comércio eletrônico, sobretudo em aspectos como desempenho, escalabilidade e experiência do usuário.

Nesse contexto, as decisões técnicas tomadas durante o desenvolvimento dessas plataformas podem influenciar diretamente os resultados das empresas. Escolhas arquiteturais, como a opção entre arquiteturas monolíticas ou baseadas em microsserviços, assim como decisões relacionadas à interface do usuário e à integração de recursos de inteligência artificial, tendem a refletir no comportamento dos usuários e, consequentemente, nas métricas de conversão e receita.

Paralelamente, a diversificação dos modelos de monetização no e-commerce tem ganhado relevância. Além do modelo tradicional de pagamento único por produto, plataformas baseadas em assinaturas recorrentes têm se consolidado em diversos segmentos, desde streaming de conteúdo até software como serviço (SaaS). Nesses modelos, métricas como receita recorrente mensal (MRR), valor do tempo de vida do cliente (CLTV) e taxa de cancelamento (\textit{churn}) tornam-se centrais para a sustentabilidade do negócio.

No mercado de e-commerce, observa-se a adoção de estratégias técnicas específicas para maximização de receita em modelos de assinatura, como: \textit{upsell}, que consiste em oferecer ao cliente um plano ou produto de maior valor durante a jornada de compra; e ofertas de retenção no momento do cancelamento (\textit{cancellation offers}), que buscam recuperar assinantes que iniciam o processo de cancelamento por meio de descontos, pausas temporárias ou alternativas de planos. Essas estratégias, embora amplamente utilizadas por empresas do setor, permanecem pouco exploradas na literatura acadêmica quanto à sua implementação técnica e impacto mensurável na receita.

\section{Motivação e Justificativa}

A motivação para este trabalho surgiu da observação do mercado de e-commerce. Empresas inseridas nesse setor frequentemente se veem diante da necessidade de implementar diversas estratégias técnicas para se manterem competitivas e aumentarem sua receita. No entanto, existe uma dificuldade prática em compreender o real impacto que cada decisão técnica pode ter nos resultados financeiros. Escolher entre diferentes arquiteturas de software, investir em otimizações de interface ou implementar sistemas de recomendação são decisões que demandam recursos, mas cujo retorno nem sempre é claro para as equipes de desenvolvimento e gestores.

Ao analisar a literatura existente, percebe-se que há trabalhos relevantes em áreas específicas. \cite{ubur_reviewing_2023} investigou o impacto da migração de arquiteturas monolíticas para microsserviços, demonstrando que cada abordagem possui vantagens em diferentes cenários de carga. No campo da experiência do usuário, \cite{muralidhar_clicks_2024} analisaram a jornada do usuário e os fatores que afetam as taxas de conversão. Na área de inteligência artificial, \cite{nguyen_personalized_2024} desenvolveram modelos de recomendação personalizada para plataformas de e-commerce.

Entretanto, esses estudos frequentemente abordam as questões de forma isolada, o que indica que há espaço para uma análise que busque integrar essas diferentes dimensões técnicas e sua relação com os resultados de negócio. Desse modo, este trabalho busca contribuir reunindo e analisando as principais estratégias técnicas encontradas na literatura e discutindo suas possíveis relações com métricas de receita e conversão. Trata-se de uma contribuição modesta, mas que pode auxiliar desenvolvedores e gestores de tecnologia a compreender melhor as opções disponíveis e suas implicações.

\section{Objetivos}

\subsection{Objetivo Geral}

Este trabalho tem como objetivo analisar o cenário atual do desenvolvimento de plataformas de e-commerce por meio de uma revisão da literatura, buscando identificar e caracterizar as principais estratégias técnicas empregadas e discutir seu potencial impacto na receita e no sucesso das empresas.

\subsection{Objetivos Específicos}

Para alcançar o objetivo geral, foram definidos os seguintes objetivos específicos:

\begin{enumerate}
    \item Identificar, por meio de revisão bibliográfica, as principais abordagens arquiteturais utilizadas em plataformas de e-commerce, como arquitetura monolítica, microsserviços e orientada a eventos, analisando seus trade-offs conforme apresentados na literatura.

    \item Examinar as estratégias de experiência do usuário (UX) e interface (UI) descritas na literatura, incluindo técnicas como Progressive Web Apps (PWA), otimização mobile e melhorias de performance.

    \item Investigar o papel da inteligência artificial e aprendizado de máquina em plataformas de e-commerce, com foco em sistemas de recomendação e personalização de conteúdo.

    \item Discutir as relações apresentadas na literatura entre decisões técnicas específicas e métricas de conversão e receita.

    \item Identificar, a partir da revisão realizada, oportunidades e direções que possam orientar trabalhos futuros na área.
\end{enumerate}

\section{Escopo e Delimitações}

Este trabalho concentra-se em quatro dimensões do desenvolvimento de plataformas de e-commerce: arquitetura de software e performance; experiência do usuário e interface; inteligência artificial aplicada; e estratégias de tráfego e conversão, incluindo aspectos de SEO técnico. A pesquisa baseia-se em revisão bibliográfica de artigos científicos, estudos de caso e documentação técnica, com foco em publicações recentes.

Além da revisão, o trabalho apresenta a aplicação prática de algumas dessas técnicas em uma plataforma real de e-commerce baseada em assinaturas, desenvolvida pelo autor. Essa aplicação prática permite ilustrar como estratégias identificadas na literatura, como otimização de performance, melhoria de experiência do usuário e técnicas de maximização de receita recorrente (\textit{upsell} e ofertas de retenção), podem ser implementadas e seus resultados observados em um contexto real de negócio.

É importante ressaltar que este trabalho não se aprofunda em questões de sazonalidade de mercado ou particularidades de tipos específicos de empresas, buscando manter uma abordagem mais geral sobre as estratégias técnicas.
% ----------------------------------------------------------------
% Revisão Bibliográfica
% ----------------------------------------------------------------
\chapter{Revisão Bibliográfica}
\label{ch:revisao}

\section{Critérios de Seleção dos Trabalhos}

Para a realização desta revisão bibliográfica, foram buscados trabalhos que abordassem aspectos técnicos do desenvolvimento de plataformas de e-commerce e sua relação com métricas de negócio. A busca foi realizada em bases como Google Scholar, ResearchGate, arXiv e repositórios de estudos de caso técnicos, utilizando termos relacionados a arquitetura de software para e-commerce, experiência do usuário em plataformas de comércio eletrônico, sistemas de recomendação, otimização de conversão e Progressive Web Apps.

Os critérios para inclusão dos trabalhos foram: (i) abordar pelo menos uma das quatro dimensões definidas no escopo deste trabalho (arquitetura e performance, UX/UI, inteligência artificial, ou tráfego e conversão); (ii) apresentar dados empíricos, estudos de caso ou análises sistemáticas; e (iii) ter relevância para o contexto de plataformas de e-commerce. Foram priorizadas publicações mais recentes, embora trabalhos anteriores tenham sido incluídos quando apresentavam contribuições relevantes para o tema.

Ao todo, foram selecionados 12 trabalhos que compõem a base desta revisão. A seguir, cada seção apresenta os principais achados organizados por eixo temático.

\section{Arquiteturas de Software e Desempenho em E-commerce}

\citeonline{zhao_systematic_2024} realizaram um mapeamento sistemático de 109 estudos que integram arquitetura de software e análise de desempenho. Os autores identificaram quatro propósitos principais nessa integração: (i) predição de desempenho baseada em modelos; (ii) detecção e resolução de anti-padrões de performance; (iii) comparação de alternativas arquiteturais; e (iv) arquiteturas auto-adaptativas para otimização dinâmica.

Os autores apontam que a maioria das pesquisas foca na fase de design, utilizando modelos como UML e PCM (\textit{Palladio Component Model}) para prever métricas de desempenho antes da implementação. Ferramentas como \textit{Apache JMeter} e \textit{SimuCom} são frequentemente utilizadas para coleta e simulação de métricas.

Entre as oportunidades identificadas por \citeonline{zhao_systematic_2024}, destacam-se: a falta de ferramentas e datasets padronizados para replicação de estudos; a necessidade de técnicas adaptadas a domínios emergentes (como sistemas blockchain e IoT); e o potencial ainda pouco explorado de técnicas de aprendizado de máquina para integrar análise arquitetural e de performance de forma mais eficiente.

No contexto específico de e-commerce, \citeonline{ubur_reviewing_2023} investigou o impacto da migração de arquiteturas monolíticas para microsserviços orientados a eventos. O autor desenvolveu protótipos de ambas as abordagens e realizou testes de carga utilizando Apache JMeter e Dropwizard para coleta de métricas.

Os resultados indicaram que aplicações monolíticas apresentam tempo de resposta mais rápido quando o número de requisições está dentro de uma faixa tolerável. Porém, à medida que a complexidade do sistema cresce e o volume de requisições aumenta, a arquitetura de microsserviços demonstra melhor desempenho. \citeonline{ubur_reviewing_2023} conclui que a escolha arquitetural deve considerar o estágio de maturidade da plataforma e a projeção de crescimento de usuários.

No âmbito de infraestrutura, \citeonline{satwika_performance_2024} avaliaram o desempenho de um website de e-commerce utilizando servidores distribuídos com balanceamento de carga Round-Robin. Os autores compararam uma configuração de servidor único (4 núcleos de CPU) com uma configuração distribuída (4 VMs com 1 núcleo cada), submetendo ambas a testes de carga com 300 usuários simultâneos.

Os resultados demonstraram que servidores distribuídos apresentam tempo de resposta 5,8 vezes mais rápido, 2,2 vezes mais respostas bem-sucedidas e capacidade de transferência de dados 2,1 vezes maior. O servidor único apresentou uma taxa de timeout 14,2 vezes maior que a configuração distribuída. Esses dados reforçam a importância de decisões de infraestrutura para plataformas que precisam lidar com alto volume de acessos simultâneos.

Os estudos apresentados nesta seção indicam que não há uma solução arquitetural universal. A escolha entre abordagens monolíticas, microsserviços ou servidores distribuídos depende de fatores como volume de tráfego esperado, recursos disponíveis e estágio de maturidade da plataforma.

\section{Experiência do Usuário e Otimizações de Interface}

\citeonline{nawir_impact_2024} investigaram a relação entre usabilidade de websites, otimização mobile, satisfação do cliente e taxas de conversão em e-commerces na Indonésia, utilizando modelagem de equações estruturais com 170 respondentes.

Os autores identificaram que a otimização mobile possui maior influência na satisfação do cliente (β = 0,489) do que a usabilidade do website (β = 0,334), refletindo o contexto de mercados onde mais de 70\% do tráfego web provém de dispositivos móveis. Elementos como design responsivo, tempo de carregamento e navegação touch-friendly foram apontados como determinantes para a experiência mobile.

Um achado relevante é que a satisfação do cliente atua como mediadora entre as otimizações técnicas e as conversões. O modelo proposto explicou 70\% da variância nas taxas de conversão, indicando que decisões técnicas relacionadas à interface e à experiência do usuário têm impacto mensurável nos resultados de vendas. Os autores destacam ainda que fatores como navegação intuitiva, clareza do conteúdo e velocidade de carregamento são elementos críticos de usabilidade que afetam diretamente a retenção de clientes.

\citeonline{jain_role_2025} apresenta uma análise dos elementos de UX/UI que influenciam conversões em e-commerce, identificando componentes-chave como navegação intuitiva, funcionalidade de busca eficiente, informações claras de produtos e processo de checkout otimizado. O autor destaca casos de sucesso: a ASOS, ao adotar uma abordagem \textit{mobile-first} com recomendações de tamanho baseadas em IA e checkout simplificado, alcançou taxa de conversão 30\% superior no mobile em comparação ao desktop.

O autor também discute tendências emergentes que estão transformando a experiência de compra online. A personalização via IA permite que algoritmos de aprendizado de máquina analisem o comportamento do usuário para entregar recomendações de produtos sob medida. A realidade aumentada (AR), utilizada por marcas como IKEA e Warby Parker, possibilita experimentação virtual de produtos, reduzindo a incerteza e aumentando a confiança na compra. \citeonline{jain_role_2025} aponta ainda o crescimento do \textit{voice commerce}, com assistentes virtuais como Alexa permitindo adicionar itens ao carrinho por comandos de voz, representando uma nova fronteira para interfaces de e-commerce.

\subsection{SEO e Performance}

\citeonline{bansal_seo_2024} investiga a relação entre práticas de SEO técnico e a velocidade de carregamento de websites, demonstrando que otimizações voltadas para mecanismos de busca também beneficiam a experiência do usuário. O autor apresenta dados que indicam uma queda de 7\% nas taxas de conversão para cada segundo adicional no tempo de carregamento de uma página.

\citeonline{bansal_seo_2024} aborda técnicas como compressão de imagens, minificação de código CSS e JavaScript, uso de cache no navegador e implementação de CDNs (\textit{Content Delivery Networks}). Um ponto relevante levantado é a importância dos Core Web Vitals do Google, que incluem métricas como LCP (\textit{Largest Contentful Paint}), FID (\textit{First Input Delay}) e CLS (\textit{Cumulative Layout Shift}). Essas métricas passaram a influenciar diretamente o ranqueamento nos resultados de busca, criando uma conexão direta entre performance técnica e visibilidade orgânica.

O autor também destaca que mais de 50\% do tráfego web global já vem de dispositivos móveis, o que torna a otimização mobile não apenas uma questão de experiência, mas também de alcance. Sites que não atendem aos critérios de velocidade tendem a sofrer penalizações no ranqueamento, reduzindo seu tráfego orgânico e, consequentemente, suas oportunidades de conversão.

\subsection{Progressive Web Apps}

Conforme documentado por \citeonline{google_alibaba_2016}, o Alibaba.com, maior plataforma de negociação B2B do mundo e presente em mais de 200 países, observou um aumento de 76\% no total de conversões em todos os navegadores após a implementação de Progressive Web App (PWA).

As PWAs combinam características de aplicativos nativos com a acessibilidade da web, oferecendo funcionalidades como carregamento offline, notificações push e instalação na tela inicial do dispositivo. Para plataformas de e-commerce, essa abordagem representa uma alternativa para melhorar a experiência do usuário mobile sem a necessidade de desenvolver aplicativos nativos separados.

O repositório \citeonline{pwa_stats_2024} documenta diversos casos de sucesso de implementação dessa tecnologia. Entre os resultados reportados por empresas de e-commerce, destacam-se: a Flipkart, que triplicou o tempo de permanência dos usuários no site; a Lancôme, com aumento de 17\% nas conversões; e a West Elm, que registrou crescimento de 15\% no tempo médio de sessão. Esses casos ilustram o potencial das PWAs em melhorar métricas de engajamento e conversão, especialmente em mercados com conexões de internet instáveis ou predominância de dispositivos móveis de entrada.

Os trabalhos revisados nesta seção convergem em um ponto: otimizações de experiência do usuário têm relação direta com métricas de receita. A queda de 7\% nas conversões por segundo adicional de carregamento \cite{bansal_seo_2024} e o aumento de 76\% nas conversões do Alibaba após implementar PWA \cite{google_alibaba_2016} exemplificam como decisões técnicas de interface podem se traduzir em resultados financeiros.

\section{Inteligência Artificial e Personalização}

\subsection{IA Generativa em E-commerce}

\citeonline{stamkou_user_2025} avaliaram a percepção de usuários sobre conteúdo gerado por inteligência artificial em uma loja virtual. Os autores desenvolveram um e-commerce de jogos de tabuleiro utilizando exclusivamente conteúdo produzido pelo ChatGPT-4 (textos, descrições de produtos) e DALL·E 3 (imagens e elementos visuais), e então aplicaram um questionário a 223 participantes que navegaram pela plataforma.

A análise fatorial identificou dois componentes principais que influenciam a experiência do usuário: ``Qualidade de Serviço e Segurança'' e ``Design e Estética''. Os resultados mostraram que o conteúdo gerado por IA foi bem recebido, com média de avaliação de 4,24 estrelas (de 5). A navegação e usabilidade tiveram aprovação superior a 88\%. Por outro lado, questões relacionadas à segurança de dados pessoais receberam avaliações mais cautelosas, indicando que a confiança ainda é um ponto sensível.

\citeonline{stamkou_user_2025} sugerem que ferramentas de IA generativa podem contribuir para o desenvolvimento de plataformas de e-commerce funcionais e visualmente atraentes, embora a transparência sobre o uso de IA e as garantias de segurança permaneçam relevantes para a aceitação dos usuários.

\subsection{Sistemas de Recomendação}

\citeonline{nguyen_personalized_2024} desenvolveram um modelo de recomendação personalizada em parceria com o grupo H\&M. O sistema combina diferentes algoritmos de recomendação, incluindo filtragem colaborativa, popularidade e \textit{Bayesian Personalized Ranking}, utilizando uma estratégia de recuperação em duas etapas.

Na primeira etapa, os algoritmos geram candidatos de produtos potencialmente relevantes. Na segunda, modelos de aprendizado de máquina avaliam e ordenam esses candidatos. Os autores compararam dois modelos: LightGBM e redes neurais profundas (\textit{Deep Neural Networks}). Os resultados mostraram que o LightGBM apresentou desempenho superior, alcançando MAP@50 de 0,06 e MAR@50 de 0,03, contra 0,02 e 0,01 respectivamente do modelo de redes neurais.

\citeonline{nguyen_personalized_2024} também abordam o problema do \textit{cold-start}, quando novos usuários ou produtos não possuem histórico suficiente para recomendações personalizadas. A combinação de múltiplas estratégias de recomendação ajuda a mitigar esse problema, permitindo que o sistema funcione mesmo com dados limitados.

\subsection{Panorama da IA em E-commerce}

\citeonline{saleh_artificial_2025} conduziram uma revisão sistemática de 21 estudos sobre aplicações de inteligência artificial em e-commerce e marketing digital. Os autores identificaram as técnicas de IA mais utilizadas: filtragem colaborativa e baseada em conteúdo (20\% cada), aprendizado de máquina (16\%), análise preditiva e processamento de linguagem natural (10\% cada).

A revisão aponta que as principais aplicações de IA no setor são: recomendações personalizadas (citadas em 10 estudos), estratégias de marketing digital (9 estudos), engajamento do cliente (8 estudos) e melhoria nas taxas de conversão (6 estudos). Os autores destacam que a IA permite análise de grandes volumes de dados em tempo real, possibilitando campanhas de marketing personalizadas e previsão de tendências de consumo.

\citeonline{saleh_artificial_2025} também identificaram desafios recorrentes na literatura: privacidade de dados (mencionada em 6 estudos), viés algorítmico (3 estudos) e necessidade de transparência nos processos decisórios baseados em IA (2 estudos). A revisão conclui que, embora a IA ofereça vantagens competitivas significativas por meio de experiências personalizadas e eficiência operacional, a integração responsável dessas tecnologias requer atenção a aspectos éticos e regulatórios.

A literatura sobre IA em e-commerce sugere que sistemas de recomendação e personalização podem influenciar positivamente as taxas de conversão. No entanto, a quantificação precisa desse impacto na receita ainda carece de estudos mais robustos, representando uma oportunidade para pesquisas futuras.

\section{Métricas de Conversão e Análise de Tráfego}

\citeonline{muralidhar_clicks_2024} realizaram uma análise de dois anos de dados do Google Analytics de uma plataforma de e-commerce, investigando a jornada do usuário desde o clique inicial até a conversão. Os autores examinaram taxas de saída e sessões por dispositivo e navegador, taxas de conversão por fonte de tráfego, e o caminho do usuário desde a visualização do produto até o checkout.

Os resultados mostraram que dispositivos móveis apresentam taxas de saída consideravelmente maiores que desktops, sugerindo problemas de otimização mobile. Entre os navegadores, Safari e Firefox tiveram taxas de saída mais altas que o Chrome, indicando possíveis incompatibilidades. Quanto às fontes de tráfego, campanhas pagas (CPM e CPC) apresentaram as melhores taxas de conversão, enquanto tráfego de referência e afiliados tiveram desempenho inferior.

Um achado relevante foi a identificação de um gargalo no funil de conversão: embora muitos usuários avancem da visualização de produtos para o checkout, apenas uma fração finaliza a compra. Isso indica que há barreiras no processo final de conversão que merecem atenção. \citeonline{muralidhar_clicks_2024} sugerem que melhorias no processo de checkout, testes A/B e otimizações específicas para mobile podem ajudar a reduzir essa distância entre interesse e transação concluída.

\section{Síntese dos Achados}

A revisão dos 12 trabalhos selecionados permite identificar convergências entre as quatro dimensões analisadas e sua relação com métricas de receita em plataformas de e-commerce.

No eixo de arquitetura, os estudos de \citeonline{ubur_reviewing_2023} e \citeonline{satwika_performance_2024} demonstram que decisões de infraestrutura afetam diretamente a capacidade de atendimento a usuários simultâneos. Embora não quantifiquem o impacto financeiro direto, a relação entre disponibilidade do sistema e oportunidades de venda é implícita: sistemas indisponíveis ou lentos resultam em vendas perdidas.

Na dimensão de experiência do usuário, a relação com receita é mais explícita. \citeonline{bansal_seo_2024} aponta queda de 7\% nas conversões por segundo adicional de carregamento, enquanto \citeonline{google_alibaba_2016} documenta aumento de 76\% nas conversões após implementação de PWA. \citeonline{nawir_impact_2024} reforçam essa conexão ao demonstrar que 70\% da variância nas taxas de conversão pode ser explicada por fatores de usabilidade e otimização mobile.

Quanto à inteligência artificial, os trabalhos revisados indicam potencial para melhoria de conversões por meio de personalização, embora os dados quantitativos sobre impacto em receita sejam menos abundantes. \citeonline{jain_role_2025} cita o caso da ASOS com 30\% mais conversões no mobile após implementar recomendações baseadas em IA, sugerindo que há espaço para estudos que quantifiquem melhor essa relação.

Por fim, a análise de \citeonline{muralidhar_clicks_2024} sobre métricas de tráfego evidencia que a identificação de gargalos no funil de conversão é essencial para direcionar investimentos técnicos de forma eficiente.

\section{Oportunidades de Pesquisa}

A partir desta revisão, identificam-se algumas oportunidades para trabalhos futuros:

\begin{itemize}
    \item Estudos que quantifiquem o retorno sobre investimento (ROI) de implementações específicas, como migração para microsserviços ou adoção de PWAs, em contextos de e-commerce brasileiro;
    \item Pesquisas que avaliem o impacto de sistemas de recomendação baseados em IA na receita de pequenas e médias empresas, considerando que a maioria dos estudos atuais foca em grandes plataformas;
    \item Investigações sobre a relação entre métricas técnicas de performance (como Core Web Vitals) e indicadores financeiros em diferentes segmentos de e-commerce.
\end{itemize}

Essas oportunidades orientam, em parte, a direção deste trabalho, que busca contribuir com uma análise integrada dessas dimensões técnicas e sua relação com resultados de negócio.

% ----------------------------------------------------------------
% Metodologia
% ----------------------------------------------------------------
\chapter{Metodologia}
\label{ch:metodologia}

Este capítulo descreve a metodologia adotada para o desenvolvimento do trabalho, que combina revisão bibliográfica com aplicação prática das estratégias identificadas.

\section{Tipo de Pesquisa}

Este trabalho caracteriza-se como pesquisa exploratória de natureza aplicada. A abordagem exploratória justifica-se pela necessidade de compreender o cenário atual das estratégias técnicas em e-commerce e suas relações com métricas de receita. A natureza aplicada decorre da intenção de não apenas mapear o conhecimento existente, mas também demonstrar a aplicação prática de algumas dessas estratégias em um contexto real de negócio.

\section{Revisão Bibliográfica}

A primeira etapa do trabalho consistiu em uma revisão bibliográfica para identificar e analisar as principais estratégias técnicas empregadas no desenvolvimento de plataformas de e-commerce.

\subsection{Critérios de Seleção}

A busca por trabalhos foi realizada em bases como Google Scholar, ResearchGate, arXiv e repositórios de estudos de caso técnicos. Os termos de busca incluíram combinações de palavras-chave relacionadas a arquitetura de software para e-commerce, experiência do usuário em plataformas de comércio eletrônico, sistemas de recomendação, otimização de conversão e Progressive Web Apps.

Os critérios para inclusão dos trabalhos foram: (i) abordar pelo menos uma das quatro dimensões definidas no escopo (arquitetura e performance, UX/UI, inteligência artificial, ou tráfego e conversão); (ii) apresentar dados empíricos, estudos de caso ou análises sistemáticas; e (iii) ter relevância para o contexto de plataformas de e-commerce. Foram priorizadas publicações recentes (2023-2025), embora trabalhos anteriores tenham sido incluídos quando apresentavam contribuições relevantes.

\subsection{Organização da Análise}

Os 12 trabalhos selecionados foram organizados em quatro eixos temáticos, permitindo uma análise estruturada das diferentes dimensões técnicas:

\begin{itemize}
    \item \textbf{Arquitetura e Performance:} estudos sobre escolhas arquiteturais (monolítico, microsserviços, servidores distribuídos) e seu impacto no desempenho;
    \item \textbf{Experiência do Usuário:} trabalhos sobre otimização de interface, performance de carregamento, SEO técnico e Progressive Web Apps;
    \item \textbf{Inteligência Artificial:} pesquisas sobre sistemas de recomendação, IA generativa e personalização;
    \item \textbf{Métricas e Conversão:} análises sobre jornada do usuário, fontes de tráfego e otimização de funil de vendas.
\end{itemize}

\section{Aplicação Prática}

Além da revisão bibliográfica, este trabalho apresenta a aplicação prática de estratégias técnicas em uma plataforma real de e-commerce. Essa etapa permite ilustrar como conceitos identificados na literatura podem ser implementados e observados em um contexto de negócio.

\subsection{Contexto da Plataforma}

A aplicação prática foi realizada na Achievece, uma plataforma de e-commerce de cursos online especializada em educação continuada para médicos nos Estados Unidos. A plataforma apresenta volume médio de aproximadamente 30 mil visitantes mensais e receita diária na ordem de US\$ 2.500. A Achievece opera com modelo de negócio híbrido, oferecendo tanto cursos com pagamento único quanto planos de assinatura recorrente, permitindo análise de estratégias específicas para ambos os modelos de monetização.

A escolha dessa plataforma justifica-se por: (i) acesso autorizado aos dados de analytics e métricas de negócio; (ii) possibilidade de implementação controlada de modificações técnicas; (iii) volume de tráfego suficiente para análise estatística significativa; (iv) representatividade do segmento de e-learning no mercado americano; e (v) diversidade de modelos de monetização que permite análise comparativa.

\subsection{Estratégias Implementadas}

Com base nas oportunidades identificadas na revisão bibliográfica e na observação de práticas do mercado, foram selecionadas frentes de aplicação prática organizadas em grupos temáticos:

\subsubsection{Otimizações de Interface e Experiência do Usuário}

A literatura revisada indica relação direta entre experiência do usuário e taxas de conversão. \citeonline{nawir_impact_2024} demonstram que 70\% da variância nas taxas de conversão pode ser explicada por fatores de usabilidade, enquanto \citeonline{muralidhar_clicks_2024} identificam gargalos entre o interesse do usuário e a conclusão da transação.

Na plataforma Achievece, são implementados testes A/B controlados utilizando a ferramenta GrowthBook para avaliar o impacto de modificações de UI/UX. As variações incluem: redesign do processo de checkout (redução de etapas, otimização de formulários), modificações na página de detalhes do produto, variações de design responsivo para dispositivos móveis e testes de copy e messaging.

\subsubsection{Progressive Web App e Otimizações de Performance}

O caso documentado do Alibaba.com, que apresentou aumento de 76\% nas conversões após implementação de PWA, motiva a análise dessas funcionalidades na plataforma. \citeonline{bansal_seo_2024} aponta queda de 7\% nas conversões para cada segundo adicional de carregamento.

São implementadas: estratégias de cache para redução de tempo de carregamento, funcionalidade offline para navegação no catálogo de cursos, notificações push para engajamento de usuários, otimização de imagens e assets (lazy loading, compressão, formatos modernos) e implementação de CDN para distribuição de conteúdo estático. O monitoramento é realizado através de Core Web Vitals (LCP, FID, CLS) via Google Lighthouse.

\subsubsection{Personalização e Recomendação}

Conforme identificado por \citeonline{nguyen_personalized_2024}, sistemas de recomendação podem impactar significativamente as métricas de conversão. Na plataforma Achievece, é implementado sistema básico de recomendação de cursos baseado em filtragem colaborativa (cursos visualizados por usuários similares), análise de navegação e comportamento, e categorização por tags de conteúdo. A avaliação é realizada via teste A/B comparando versões com e sem recomendações personalizadas.

\subsubsection{SEO e Análise de Fontes de Tráfego}

\citeonline{bansal_seo_2024} destaca a relação entre SEO técnico, velocidade de carregamento e receita. São implementadas otimizações de meta tags, títulos e descrições, melhoria de estrutura de URLs e breadcrumbs, e implementação de schema markup para rich snippets.

Adicionalmente, seguindo a análise de \citeonline{muralidhar_clicks_2024} sobre fontes de tráfego, é realizada análise comparativa do comportamento e performance de diferentes canais: tráfego orgânico, pago, direto, de referência e social.

\subsubsection{Estratégias de Maximização de Receita Recorrente}

Além das estratégias identificadas na literatura, a observação do mercado de e-commerce revela práticas amplamente adotadas por empresas com modelo de assinatura que carecem de estudos acadêmicos sobre sua implementação técnica e efetividade:

\textbf{Upsell:} implementação de ofertas de upgrade de plano durante a jornada de compra (checkout) e pós-compra (para assinantes ativos). O objetivo é observar a taxa de aceitação do upgrade, o timing ótimo para apresentação de ofertas e o impacto na receita média por transação.

\textbf{Ofertas de retenção (cancellation offers):} implementação de fluxo de cancelamento que oferece alternativas ao usuário que inicia o processo de cancelamento, incluindo descontos temporários, pausa da assinatura ou downgrade para plano de menor valor. O objetivo é observar a taxa de retenção de assinantes e o impacto no churn mensal.

\textbf{Recuperação de carrinho abandonado:} implementação de emails automatizados de lembrete com variações de timing (1h, 24h, 48h após abandono) e testes de copy e ofertas.

\subsection{Coleta e Análise de Dados}

Os dados são coletados por meio de ferramentas de analytics integradas à plataforma:

\begin{itemize}
    \item Google Analytics 4 para análise de comportamento e conversões
    \item Google Lighthouse para métricas de performance e Core Web Vitals
    \item Google Search Console para dados de tráfego orgânico e SEO
    \item GrowthBook para execução e análise de testes A/B
    \item Lucky Orange para análise comportamental (heatmaps, session recordings, funnels)
\end{itemize}

As métricas observadas incluem:

\begin{itemize}
    \item \textbf{Performance:} tempo de carregamento (LCP), interatividade (FID), estabilidade visual (CLS);
    \item \textbf{Conversão:} taxa de conversão por etapa do funil, taxa de abandono de carrinho, taxa de conclusão de checkout, taxa de recuperação de carrinho;
    \item \textbf{Receita:} valor médio do pedido (AOV), receita por visitante (RPV), receita total;
    \item \textbf{Assinaturas:} MRR, taxa de churn, CLTV, ARPU, taxa de conversão de upsell, taxa de retenção via cancellation offers.
\end{itemize}

A análise estatística utiliza testes de significância (teste t para médias, teste qui-quadrado para proporções) com nível de confiança de 95\%. Os testes A/B são conduzidos com período mínimo de 2 semanas por experimento para garantir significância estatística considerando o volume de tráfego da plataforma.

\section{Limitações Metodológicas}

É importante reconhecer as limitações desta abordagem metodológica:

\begin{itemize}
    \item A aplicação prática foi realizada em uma única plataforma, com características específicas (e-commerce de cursos online, segmento de educação médica nos EUA), o que limita a generalização dos resultados para outros contextos;
    \item O período de observação pode não capturar completamente efeitos de longo prazo ou variações anuais;
    \item Fatores externos (sazonalidade do mercado educacional, campanhas de marketing, mudanças na concorrência, variações macroeconômicas) podem influenciar as métricas observadas, dificultando o isolamento do impacto de cada estratégia implementada;
    \item O volume de tráfego, embora significativo, pode limitar a detecção de efeitos pequenos em certos pontos de observação.
\end{itemize}

Essas limitações são inerentes a estudos de aplicação prática em ambientes reais de negócio. Os resultados apresentados devem ser interpretados como observações ilustrativas das estratégias identificadas na literatura e no mercado, não como evidências definitivas de relações causais.


% ----------------------------------------------------------------
% Análise e Discussão
% ----------------------------------------------------------------
\chapter{Análise e Discussão}
\label{ch:analise}

% TODO: Este capítulo será desenvolvido posteriormente

\section{Panorama Geral das Estratégias Técnicas}

\section{Relação entre Arquitetura e Performance}

\subsection{Comparação de Abordagens Arquiteturais}

\subsection{Impacto na Escalabilidade}

\section{UX/UI e Conversões: Análise Integrada}

\subsection{Fatores Críticos de Sucesso}

\subsection{PWA: Caso de Sucesso Alibaba.com}

\subsection{Mobile como Prioridade}

\section{Inteligência Artificial e ROI}

\subsection{Sistemas de Recomendação: Análise Crítica}

\subsection{IA Generativa: Potencial e Limitações}

\subsection{Personalização e Aumento de Receita}

\section{Métricas de Conversão: Insights e Padrões}

\subsection{Funil de Conversão Otimizado}

\subsection{Abandono de Carrinho}

\section{Síntese: Decisões Técnicas x Impacto Financeiro}

% ----------------------------------------------------------------
% Conclusão
% ----------------------------------------------------------------
\chapter{Conclusão}
\label{ch:conclusao}

Este capítulo apresenta as conclusões do trabalho, retomando os objetivos propostos e discutindo em que medida foram alcançados. São apresentadas também as limitações identificadas durante a pesquisa e sugestões para trabalhos futuros.

\section{Retomada dos Objetivos}

O objetivo geral deste trabalho foi analisar o cenário atual do desenvolvimento de plataformas de e-commerce, buscando identificar estratégias técnicas e discutir seu potencial impacto na receita das empresas. Para isso, foram estabelecidos cinco objetivos específicos, cujo alcance é discutido a seguir.

O primeiro objetivo, de identificar abordagens arquiteturais utilizadas em e-commerce, foi parcialmente atendido. A revisão bibliográfica permitiu mapear discussões sobre arquiteturas monolíticas versus distribuídas \cite{ubur_reviewing_2023}. A aplicação prática na Achievece ilustrou a migração de WordPress para Next.js com arquitetura serverless, com melhorias na estabilidade. No entanto, a literatura apresentou poucos estudos quantitativos relacionando escolhas arquiteturais com métricas de receita.

O segundo objetivo, de examinar estratégias de UX/UI, foi atendido de forma satisfatória. Foram identificados estudos que relacionam usabilidade com taxas de conversão, indicando que 70\% da variância nas conversões pode ser explicada por fatores de usabilidade \cite{nawir_impact_2024}. Na aplicação prática, a reorganização da interface resultou em aumento de 74\% no AOV de assinaturas e crescimento de 105\% na participação de novas vendas dessa categoria.

O terceiro objetivo, de investigar o papel da IA em e-commerce, foi parcialmente atendido. A revisão identificou trabalhos sobre sistemas de recomendação \cite{nguyen_personalized_2024} e sobre IA generativa aplicada a e-commerce. Na Achievece, a personalização foi implementada de forma manual, sem uso de algoritmos de IA, o que limitou a análise prática desse objetivo.

O quarto objetivo, de discutir relações entre decisões técnicas e métricas de receita, foi o mais desafiador. A literatura revisada apresenta evidências fragmentadas dessa relação, e a aplicação prática enfrentou dificuldades de atribuição devido à implementação simultânea de múltiplas estratégias. Os dados observados sugerem correlação entre o conjunto de mudanças e melhorias nas métricas, mas não permitem estabelecer causalidade isolada.

O quinto objetivo, de identificar oportunidades para trabalhos futuros, foi atendido e é detalhado na seção correspondente deste capítulo.

\section{Principais Observações}

A partir da revisão bibliográfica e da aplicação prática, algumas observações podem ser destacadas:

\begin{itemize}
    \item A estabilidade da plataforma, após a migração para arquitetura serverless, mostrou-se pré-requisito para qualquer estratégia de conversão. Uma plataforma indisponível não converte.

    \item As métricas de performance (Core Web Vitals) alcançadas foram satisfatórias, com score de 95 no Vercel Speed Insights. A ausência de dados históricos impediu uma análise comparativa.

    \item A reorganização da interface, posicionando produtos de assinatura em destaque, coincidiu com aumento de 74\% no AOV dessa categoria e crescimento de 105\% na participação de novas vendas. O AOV geral da plataforma aumentou 17\%.

    \item As otimizações de SEO técnico, incluindo dados estruturados JSON-LD, coincidiram com aumento de 102\% no tráfego proveniente de plataformas de IA (chatgpt.com) entre julho e agosto de 2025.

    \item A estratégia de ofertas de retenção no fluxo de cancelamento apresentou resultado mensurável: redução de 30\% na intenção de cancelamento. Por afetar um momento específico da jornada do usuário, essa foi a estratégia com atribuição mais clara.

    \item A dificuldade de atribuição de resultados a estratégias específicas confirmou-se como desafio central em estudos realizados em ambientes de produção.
\end{itemize}

\section{Limitações do Estudo}

Este trabalho possui limitações que devem ser consideradas na interpretação dos resultados:

A principal limitação é a impossibilidade de isolar o impacto de cada estratégia. Por se tratar de uma empresa em funcionamento, as mudanças foram implementadas em períodos próximos, sem grupo de controle. Não foi possível aplicar testes A/B devido a questões operacionais da empresa, o que impede afirmações de causalidade entre estratégias específicas e resultados observados.

A perda de dados históricos da plataforma anterior em WordPress limitou a análise comparativa, especialmente para métricas de performance. Sem linha de base, não é possível quantificar a melhoria proporcionada pela nova arquitetura.

Os recortes temporais apresentados, embora selecionados de um período de aproximadamente um ano de desenvolvimento, podem ser insuficientes para capturar efeitos de longo prazo ou variações sazonais. Resultados observados podem não se sustentar ao longo do tempo.

A amostra de um único caso (plataforma Achievece) limita a generalização dos achados. Plataformas de outros segmentos, tamanhos ou públicos podem apresentar comportamentos distintos.

Por fim, a revisão bibliográfica, embora tenha buscado abranger trabalhos relevantes, não seguiu protocolo formal de revisão sistemática, o que pode ter resultado em viés de seleção dos artigos analisados.

\section{Trabalhos Futuros}

Com base nas lacunas identificadas na literatura e nas limitações deste estudo, sugerem-se as seguintes direções para trabalhos futuros:

\textbf{Estudos com grupos de controle}: Pesquisas que implementem testes A/B para isolar o impacto de estratégias específicas em métricas de conversão e receita.

\textbf{Análises longitudinais}: Estudos que acompanhem plataformas por períodos mais longos, capturando efeitos de sazonalidade e permitindo observar a sustentabilidade dos resultados ao longo do tempo.

\textbf{Comparações entre segmentos}: Pesquisas que comparem o impacto de estratégias em diferentes tipos de e-commerce (B2B, B2C, assinaturas, marketplace).

\textbf{Frameworks de atribuição}: Desenvolvimento de metodologias para atribuir resultados a estratégias específicas em ambientes com múltiplas variáveis.

\textbf{Integração de IA em personalização}: Estudos que comparem personalização manual versus automatizada por algoritmos de IA.

\textbf{Ambientes controlados em empresas reais}: Uma direção relevante para trabalhos futuros seria a criação de condições que permitam maior controle experimental dentro de empresas em funcionamento. Isso exigiria: (i) alocação de recursos específicos da empresa para pesquisa, separando iniciativas experimentais das demandas operacionais do dia a dia; (ii) definição de períodos dedicados à análise, evitando que decisões de negócio urgentes interfiram no cronograma de observação; (iii) implementação de infraestrutura para testes A/B desde o início do projeto, e não como adaptação posterior; e (iv) comprometimento da liderança em priorizar a coleta de dados mesmo quando isso implique atrasar lançamentos de funcionalidades. Este trabalho evidenciou que, em ambientes onde a operação comercial é prioritária, as demandas de curto prazo frequentemente comprometem o rigor metodológico. Considerando que os resultados observados sugerem impacto positivo nas métricas de receita, investir em estrutura para estudos mais controlados pode se justificar financeiramente, uma vez que permitiria identificar com precisão quais estratégias geram maior retorno e direcionar recursos de forma mais eficiente.

\section{Considerações Finais}

Este trabalho buscou contribuir para a compreensão das relações entre estratégias técnicas e resultados de negócio em plataformas de e-commerce. É uma contribuição modesta, que combina revisão bibliográfica com observações de uma aplicação prática.

Os resultados na Achievece, embora limitados pelas dificuldades de atribuição discutidas, sugerem que o conjunto de estratégias implementadas contribuiu para melhorias nas métricas de receita, com destaque para o crescimento nas vendas de produtos por assinatura.

A principal contribuição deste trabalho é mostrar a complexidade envolvida na análise do impacto de decisões técnicas em métricas de negócio. A literatura oferece evidências fragmentadas, e a prática empresarial raramente permite o isolamento de variáveis necessário para conclusões definitivas.

Para desenvolvedores e gestores de tecnologia, espera-se que este trabalho ofereça um panorama das estratégias disponíveis e suas possíveis implicações.




% ----------------------------------------------------------
% ELEMENTOS PÓS-TEXTUAIS
% ----------------------------------------------------------
\postextual
% ----------------------------------------------------------

% ----------------------------------------------------------
% Referências bibliográficas
% ----------------------------------------------------------
\bibliography{referencias_tcc}

% ----------------------------------------------------------
% Glossário
% ----------------------------------------------------------
%
% Consulte o manual da classe abntex2 para orientações sobre o glossário.
%
%\glossary

% ----------------------------------------------------------
% Apêndices (se necessário)
% ----------------------------------------------------------
% Descomente a seção abaixo se houver apêndices a incluir
%
% \begin{apendicesenv}
% \partapendices
% \chapter{Título do Apêndice A}
% Conteúdo do apêndice A
% \end{apendicesenv}
% ---



\end{document}
