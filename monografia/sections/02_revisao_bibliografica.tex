% ----------------------------------------------------------------
% Revisão Bibliográfica
% ----------------------------------------------------------------
\chapter{Revisão Bibliográfica}
\label{ch:revisao}

\section{Critérios de Seleção dos Trabalhos}

Para a realização desta revisão bibliográfica, foram buscados trabalhos que abordassem aspectos técnicos do desenvolvimento de plataformas de e-commerce e sua relação com métricas de negócio. A busca foi realizada em bases como Google Scholar, ResearchGate, arXiv e repositórios de estudos de caso técnicos, utilizando termos relacionados a arquitetura de software para e-commerce, experiência do usuário em plataformas de comércio eletrônico, sistemas de recomendação, otimização de conversão e Progressive Web Apps.

Os critérios para inclusão dos trabalhos foram: (i) abordar pelo menos uma das quatro dimensões definidas no escopo deste trabalho (arquitetura e performance, UX/UI, inteligência artificial, ou tráfego e conversão); (ii) apresentar dados empíricos, estudos de caso ou análises sistemáticas; e (iii) ter relevância para o contexto de plataformas de e-commerce. Foram priorizadas publicações mais recentes, embora trabalhos anteriores tenham sido incluídos quando apresentavam contribuições relevantes para o tema.

Ao todo, foram selecionados 12 trabalhos que compõem a base desta revisão. A seguir, cada seção apresenta os principais achados organizados por eixo temático.

\section{Arquiteturas de Software e Desempenho em E-commerce}

\citeonline{zhao_systematic_2024} realizaram um mapeamento sistemático de 109 estudos que integram arquitetura de software e análise de desempenho. Os autores identificaram quatro propósitos principais nessa integração: (i) predição de desempenho baseada em modelos; (ii) detecção e resolução de anti-padrões de performance; (iii) comparação de alternativas arquiteturais; e (iv) arquiteturas auto-adaptativas para otimização dinâmica.

Os autores apontam que a maioria das pesquisas foca na fase de design, utilizando modelos como UML e PCM (\textit{Palladio Component Model}) para prever métricas de desempenho antes da implementação. Ferramentas como \textit{Apache JMeter} e \textit{SimuCom} são frequentemente utilizadas para coleta e simulação de métricas.

Entre as oportunidades identificadas por \citeonline{zhao_systematic_2024}, destacam-se: a falta de ferramentas e datasets padronizados para replicação de estudos; a necessidade de técnicas adaptadas a domínios emergentes (como sistemas blockchain e IoT); e o potencial ainda pouco explorado de técnicas de aprendizado de máquina para integrar análise arquitetural e de performance de forma mais eficiente.

No contexto específico de e-commerce, \citeonline{ubur_reviewing_2023} investigou o impacto da migração de arquiteturas monolíticas para microsserviços orientados a eventos. O autor desenvolveu protótipos de ambas as abordagens e realizou testes de carga utilizando Apache JMeter e Dropwizard para coleta de métricas.

Os resultados indicaram que aplicações monolíticas apresentam tempo de resposta mais rápido quando o número de requisições está dentro de uma faixa tolerável. Porém, à medida que a complexidade do sistema cresce e o volume de requisições aumenta, a arquitetura de microsserviços demonstra melhor desempenho. \citeonline{ubur_reviewing_2023} conclui que a escolha arquitetural deve considerar o estágio de maturidade da plataforma e a projeção de crescimento de usuários.

No âmbito de infraestrutura, \citeonline{satwika_performance_2024} avaliaram o desempenho de um website de e-commerce utilizando servidores distribuídos com balanceamento de carga Round-Robin. Os autores compararam uma configuração de servidor único (4 núcleos de CPU) com uma configuração distribuída (4 VMs com 1 núcleo cada), submetendo ambas a testes de carga com 300 usuários simultâneos.

Os resultados demonstraram que servidores distribuídos apresentam tempo de resposta 5,8 vezes mais rápido, 2,2 vezes mais respostas bem-sucedidas e capacidade de transferência de dados 2,1 vezes maior. O servidor único apresentou uma taxa de timeout 14,2 vezes maior que a configuração distribuída. Esses dados reforçam a importância de decisões de infraestrutura para plataformas que precisam lidar com alto volume de acessos simultâneos.

Os estudos apresentados nesta seção indicam que não há uma solução arquitetural universal. A escolha entre abordagens monolíticas, microsserviços ou servidores distribuídos depende de fatores como volume de tráfego esperado, recursos disponíveis e estágio de maturidade da plataforma.

\section{Experiência do Usuário e Otimizações de Interface}

\citeonline{nawir_impact_2024} investigaram a relação entre usabilidade de websites, otimização mobile, satisfação do cliente e taxas de conversão em e-commerces na Indonésia, utilizando modelagem de equações estruturais com 170 respondentes.

Os autores identificaram que a otimização mobile possui maior influência na satisfação do cliente (β = 0,489) do que a usabilidade do website (β = 0,334), refletindo o contexto de mercados onde mais de 70\% do tráfego web provém de dispositivos móveis. Elementos como design responsivo, tempo de carregamento e navegação touch-friendly foram apontados como determinantes para a experiência mobile.

Um achado relevante é que a satisfação do cliente atua como mediadora entre as otimizações técnicas e as conversões. O modelo proposto explicou 70\% da variância nas taxas de conversão, indicando que decisões técnicas relacionadas à interface e à experiência do usuário têm impacto mensurável nos resultados de vendas. Os autores destacam ainda que fatores como navegação intuitiva, clareza do conteúdo e velocidade de carregamento são elementos críticos de usabilidade que afetam diretamente a retenção de clientes.

\citeonline{jain_role_2025} apresenta uma análise dos elementos de UX/UI que influenciam conversões em e-commerce, identificando componentes-chave como navegação intuitiva, funcionalidade de busca eficiente, informações claras de produtos e processo de checkout otimizado. O autor destaca casos de sucesso: a ASOS, ao adotar uma abordagem \textit{mobile-first} com recomendações de tamanho baseadas em IA e checkout simplificado, alcançou taxa de conversão 30\% superior no mobile em comparação ao desktop.

O autor também discute tendências emergentes que estão transformando a experiência de compra online. A personalização via IA permite que algoritmos de aprendizado de máquina analisem o comportamento do usuário para entregar recomendações de produtos sob medida. A realidade aumentada (AR), utilizada por marcas como IKEA e Warby Parker, possibilita experimentação virtual de produtos, reduzindo a incerteza e aumentando a confiança na compra. \citeonline{jain_role_2025} aponta ainda o crescimento do \textit{voice commerce}, com assistentes virtuais como Alexa permitindo adicionar itens ao carrinho por comandos de voz, representando uma nova fronteira para interfaces de e-commerce.

\subsection{SEO e Performance}

\citeonline{bansal_seo_2024} investiga a relação entre práticas de SEO técnico e a velocidade de carregamento de websites, demonstrando que otimizações voltadas para mecanismos de busca também beneficiam a experiência do usuário. O autor apresenta dados que indicam uma queda de 7\% nas taxas de conversão para cada segundo adicional no tempo de carregamento de uma página.

\citeonline{bansal_seo_2024} aborda técnicas como compressão de imagens, minificação de código CSS e JavaScript, uso de cache no navegador e implementação de CDNs (\textit{Content Delivery Networks}). Um ponto relevante levantado é a importância dos Core Web Vitals do Google, que incluem métricas como LCP (\textit{Largest Contentful Paint}), FID (\textit{First Input Delay}) e CLS (\textit{Cumulative Layout Shift}). Essas métricas passaram a influenciar diretamente o ranqueamento nos resultados de busca, criando uma conexão direta entre performance técnica e visibilidade orgânica.

O autor também destaca que mais de 50\% do tráfego web global já vem de dispositivos móveis, o que torna a otimização mobile não apenas uma questão de experiência, mas também de alcance. Sites que não atendem aos critérios de velocidade tendem a sofrer penalizações no ranqueamento, reduzindo seu tráfego orgânico e, consequentemente, suas oportunidades de conversão.

\subsection{Progressive Web Apps}

Conforme documentado por \citeonline{google_alibaba_2016}, o Alibaba.com, maior plataforma de negociação B2B do mundo e presente em mais de 200 países, observou um aumento de 76\% no total de conversões em todos os navegadores após a implementação de Progressive Web App (PWA).

As PWAs combinam características de aplicativos nativos com a acessibilidade da web, oferecendo funcionalidades como carregamento offline, notificações push e instalação na tela inicial do dispositivo. Para plataformas de e-commerce, essa abordagem representa uma alternativa para melhorar a experiência do usuário mobile sem a necessidade de desenvolver aplicativos nativos separados.

O repositório \citeonline{pwa_stats_2024} documenta diversos casos de sucesso de implementação dessa tecnologia. Entre os resultados reportados por empresas de e-commerce, destacam-se: a Flipkart, que triplicou o tempo de permanência dos usuários no site; a Lancôme, com aumento de 17\% nas conversões; e a West Elm, que registrou crescimento de 15\% no tempo médio de sessão. Esses casos ilustram o potencial das PWAs em melhorar métricas de engajamento e conversão, especialmente em mercados com conexões de internet instáveis ou predominância de dispositivos móveis de entrada.

Os trabalhos revisados nesta seção convergem em um ponto: otimizações de experiência do usuário têm relação direta com métricas de receita. A queda de 7\% nas conversões por segundo adicional de carregamento \cite{bansal_seo_2024} e o aumento de 76\% nas conversões do Alibaba após implementar PWA \cite{google_alibaba_2016} exemplificam como decisões técnicas de interface podem se traduzir em resultados financeiros.

\section{Inteligência Artificial e Personalização}

\subsection{IA Generativa em E-commerce}

\citeonline{stamkou_user_2025} avaliaram a percepção de usuários sobre conteúdo gerado por inteligência artificial em uma loja virtual. Os autores desenvolveram um e-commerce de jogos de tabuleiro utilizando exclusivamente conteúdo produzido pelo ChatGPT-4 (textos, descrições de produtos) e DALL·E 3 (imagens e elementos visuais), e então aplicaram um questionário a 223 participantes que navegaram pela plataforma.

A análise fatorial identificou dois componentes principais que influenciam a experiência do usuário: ``Qualidade de Serviço e Segurança'' e ``Design e Estética''. Os resultados mostraram que o conteúdo gerado por IA foi bem recebido, com média de avaliação de 4,24 estrelas (de 5). A navegação e usabilidade tiveram aprovação superior a 88\%. Por outro lado, questões relacionadas à segurança de dados pessoais receberam avaliações mais cautelosas, indicando que a confiança ainda é um ponto sensível.

\citeonline{stamkou_user_2025} sugerem que ferramentas de IA generativa podem contribuir para o desenvolvimento de plataformas de e-commerce funcionais e visualmente atraentes, embora a transparência sobre o uso de IA e as garantias de segurança permaneçam relevantes para a aceitação dos usuários.

\subsection{Sistemas de Recomendação}

\citeonline{nguyen_personalized_2024} desenvolveram um modelo de recomendação personalizada em parceria com o grupo H\&M. O sistema combina diferentes algoritmos de recomendação, incluindo filtragem colaborativa, popularidade e \textit{Bayesian Personalized Ranking}, utilizando uma estratégia de recuperação em duas etapas.

Na primeira etapa, os algoritmos geram candidatos de produtos potencialmente relevantes. Na segunda, modelos de aprendizado de máquina avaliam e ordenam esses candidatos. Os autores compararam dois modelos: LightGBM e redes neurais profundas (\textit{Deep Neural Networks}). Os resultados mostraram que o LightGBM apresentou desempenho superior, alcançando MAP@50 de 0,06 e MAR@50 de 0,03, contra 0,02 e 0,01 respectivamente do modelo de redes neurais.

\citeonline{nguyen_personalized_2024} também abordam o problema do \textit{cold-start}, quando novos usuários ou produtos não possuem histórico suficiente para recomendações personalizadas. A combinação de múltiplas estratégias de recomendação ajuda a mitigar esse problema, permitindo que o sistema funcione mesmo com dados limitados.

\subsection{Panorama da IA em E-commerce}

\citeonline{saleh_artificial_2025} conduziram uma revisão sistemática de 21 estudos sobre aplicações de inteligência artificial em e-commerce e marketing digital. Os autores identificaram as técnicas de IA mais utilizadas: filtragem colaborativa e baseada em conteúdo (20\% cada), aprendizado de máquina (16\%), análise preditiva e processamento de linguagem natural (10\% cada).

A revisão aponta que as principais aplicações de IA no setor são: recomendações personalizadas (citadas em 10 estudos), estratégias de marketing digital (9 estudos), engajamento do cliente (8 estudos) e melhoria nas taxas de conversão (6 estudos). Os autores destacam que a IA permite análise de grandes volumes de dados em tempo real, possibilitando campanhas de marketing personalizadas e previsão de tendências de consumo.

\citeonline{saleh_artificial_2025} também identificaram desafios recorrentes na literatura: privacidade de dados (mencionada em 6 estudos), viés algorítmico (3 estudos) e necessidade de transparência nos processos decisórios baseados em IA (2 estudos). A revisão conclui que, embora a IA ofereça vantagens competitivas significativas por meio de experiências personalizadas e eficiência operacional, a integração responsável dessas tecnologias requer atenção a aspectos éticos e regulatórios.

A literatura sobre IA em e-commerce sugere que sistemas de recomendação e personalização podem influenciar positivamente as taxas de conversão. No entanto, a quantificação precisa desse impacto na receita ainda carece de estudos mais robustos, representando uma oportunidade para pesquisas futuras.

\section{Métricas de Conversão e Análise de Tráfego}

\citeonline{muralidhar_clicks_2024} realizaram uma análise de dois anos de dados do Google Analytics de uma plataforma de e-commerce, investigando a jornada do usuário desde o clique inicial até a conversão. Os autores examinaram taxas de saída e sessões por dispositivo e navegador, taxas de conversão por fonte de tráfego, e o caminho do usuário desde a visualização do produto até o checkout.

Os resultados mostraram que dispositivos móveis apresentam taxas de saída consideravelmente maiores que desktops, sugerindo problemas de otimização mobile. Entre os navegadores, Safari e Firefox tiveram taxas de saída mais altas que o Chrome, indicando possíveis incompatibilidades. Quanto às fontes de tráfego, campanhas pagas (CPM e CPC) apresentaram as melhores taxas de conversão, enquanto tráfego de referência e afiliados tiveram desempenho inferior.

Um achado relevante foi a identificação de um gargalo no funil de conversão: embora muitos usuários avancem da visualização de produtos para o checkout, apenas uma fração finaliza a compra. Isso indica que há barreiras no processo final de conversão que merecem atenção. \citeonline{muralidhar_clicks_2024} sugerem que melhorias no processo de checkout, testes A/B e otimizações específicas para mobile podem ajudar a reduzir essa distância entre interesse e transação concluída.

\section{Síntese dos Achados}

A revisão dos 12 trabalhos selecionados permite identificar convergências entre as quatro dimensões analisadas e sua relação com métricas de receita em plataformas de e-commerce.

No eixo de arquitetura, os estudos de \citeonline{ubur_reviewing_2023} e \citeonline{satwika_performance_2024} demonstram que decisões de infraestrutura afetam diretamente a capacidade de atendimento a usuários simultâneos. Embora não quantifiquem o impacto financeiro direto, a relação entre disponibilidade do sistema e oportunidades de venda é implícita: sistemas indisponíveis ou lentos resultam em vendas perdidas.

Na dimensão de experiência do usuário, a relação com receita é mais explícita. \citeonline{bansal_seo_2024} aponta queda de 7\% nas conversões por segundo adicional de carregamento, enquanto \citeonline{google_alibaba_2016} documenta aumento de 76\% nas conversões após implementação de PWA. \citeonline{nawir_impact_2024} reforçam essa conexão ao demonstrar que 70\% da variância nas taxas de conversão pode ser explicada por fatores de usabilidade e otimização mobile.

Quanto à inteligência artificial, os trabalhos revisados indicam potencial para melhoria de conversões por meio de personalização, embora os dados quantitativos sobre impacto em receita sejam menos abundantes. \citeonline{jain_role_2025} cita o caso da ASOS com 30\% mais conversões no mobile após implementar recomendações baseadas em IA, sugerindo que há espaço para estudos que quantifiquem melhor essa relação.

Por fim, a análise de \citeonline{muralidhar_clicks_2024} sobre métricas de tráfego evidencia que a identificação de gargalos no funil de conversão é essencial para direcionar investimentos técnicos de forma eficiente.

\section{Oportunidades de Pesquisa}

A partir desta revisão, identificam-se algumas oportunidades para trabalhos futuros:

\begin{itemize}
    \item Estudos que quantifiquem o retorno sobre investimento (ROI) de implementações específicas, como migração para microsserviços ou adoção de PWAs, em contextos de e-commerce brasileiro;
    \item Pesquisas que avaliem o impacto de sistemas de recomendação baseados em IA na receita de pequenas e médias empresas, considerando que a maioria dos estudos atuais foca em grandes plataformas;
    \item Investigações sobre a relação entre métricas técnicas de performance (como Core Web Vitals) e indicadores financeiros em diferentes segmentos de e-commerce.
\end{itemize}

Essas oportunidades orientam, em parte, a direção deste trabalho, que busca contribuir com uma análise integrada dessas dimensões técnicas e sua relação com resultados de negócio.
